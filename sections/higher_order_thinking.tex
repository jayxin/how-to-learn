
\documentclass[../main.tex]{subfiles}
\graphicspath{{\subfix{../figures/}}}
%
\title{Higher Order Thinking: Bloom's Taxonomy}
%
\begin{document}
\maketitle
\section{Introduction}
Many students start college using the study strategies they used in high school,
which is understandable—the strategies worked in the past, so why wouldn't they
work now? As you may have already figured out, college is different. Classes may
be more rigorous (yet may seem less structured), your reading load may be
heavier, and your professors may be less accessible. For these reasons and
others, you'll likely find that your old study habits aren't as effective as
they used to be. Part of the reason for this is that you may not be approaching
the material in the same way as your professors. In this handout, we provide
information on Bloom's Taxonomy—a way of thinking about your schoolwork that can
change the way you study and learn to better align with how your professors
think (and how they grade).
%
\section{Why higher order thinking leads to effective study}
Most students report that high school was largely about remembering and
understanding large amounts of content and then demonstrating this comprehension
periodically on tests and exams. Bloom's Taxonomy is a framework that starts
with these two levels of thinking as important bases for pushing our brains to
five other higher order levels of thinking—helping us move beyond remembering
and recalling information and move deeper into application, analysis, synthesis,
evaluation, and creation—the levels of thinking that your professors have in
mind when they are designing exams and paper assignments. Because it is in these
higher levels of thinking that our brains truly and deeply learn information,
it's important that you integrate higher order thinking into your study habits.

The following categories can help you assess your comprehension of readings,
lecture notes, and other course materials. By creating and answering questions
from a variety of categories, you can better anticipate and prepare for all
types of exam questions. As you learn and study, start by asking yourself
questions and using study methods from the level of remembering. Then, move
progressively through the levels to push your understanding deeper—making your
studying more meaningful and improving your long-term retention.
%
\subsection{Level 1: Remember}
This level helps us recall foundational or factual information: names, dates,
formulas, definitions, components, or methods.
%
\begin{table}[H]
  \begin{center}
    \begin{tabular}[c]{p{6cm}p{6cm}}
      \hline
      Study methods & Types of questions to ask yourself \\
      \hline
      Make and use flashcards for key terms. & How would you define $ \ldots $?
      \\
      Make a list or timeline of the main events. & List the $ \ldots $ in
      order. \\
      List the main characteristics of something. & Who were $ \ldots $? \\
      \hline
    \end{tabular}
    \caption{Level 1: Remember}
  \end{center}
\end{table}
%
\subsection{Level 2: Understand}
Understanding means that we can explain main ideas and concepts and make meaning
by interpreting, classifying, summarizing, inferring, comparing, and explaining.
%
\begin{table}[H]
  \begin{center}
    \begin{tabular}[c]{p{6cm}p{6cm}}
      \hline
      Study methods & Types of questions to ask yourself \\
      \hline
      Discuss content with or explain to a partner. & How would you
      differentiate between $ \ldots $ and $ \ldots $? \\
      Explain the main idea of the section. & What is the main idea of $ \ldots
      $? \\
      Write a summary of the chapter in your own words. & Why did $ \ldots $? \\
      \hline
    \end{tabular}
    \caption{Level 2: Understand}
  \end{center}
\end{table}
%
\subsection{Level 3: Apply}
Application allows us to recognize or use concepts in real-world situations and
to address when, where, or how to employ methods and ideas.
%
\begin{table}[H]
  \begin{center}
    \begin{tabular}[c]{p{6cm}p{6cm}}
      \hline
      Study methods & Types of questions to ask yourself \\
      \hline
      Seek concrete examples of abstract ideas. &	Why does $ \ldots $ work? \\
      Work practice problems and exercises. &	How would you change $ \ldots $?
      \\
      Write an instructional manual or study guide on the chapter that others
      could use. &	How would you develop a set of instructions about $ \ldots
      $? \\
      \hline
    \end{tabular}
    \caption{Level 3: Apply}
  \end{center}
\end{table}
%
\subsection{Level 4: Analyze}
Analysis means breaking a topic or idea into components or examining a subject
from different perspectives. It helps us see how the ``whole'' is created from the
``parts''. It's easy to miss the big picture by getting stuck at a lower level of
thinking and simply remembering individual facts without seeing how they are
connected. Analysis helps reveal the connections between facts.
%
\begin{table}[H]
  \begin{center}
    \begin{tabular}[c]{p{6cm}p{6cm}}
      \hline
      Study methods & Types of questions to ask yourself \\
      \hline
      Generate a list of contributing factors. & How does this element
      contribute to the whole? \\
      Determine the importance of different elements or sections. &	What is the
      significance of this section? \\
      Think about it from a different perspective. & How would $ \ldots $ group
      see this? \\
      \hline
    \end{tabular}
    \caption{Level 4: Analyze}
  \end{center}
\end{table}
%
\subsection{Level 5: Synthesize}
Synthesizing means considering individual elements together for the purpose of
drawing conclusions, identifying themes, or determining common elements. Here
you want to shift from ``parts'' to ``whole''.
%
\begin{table}[H]
  \begin{center}
    \begin{tabular}[c]{p{6cm}p{6cm}}
      \hline
      Study methods & Types of questions to ask yourself \\
      \hline
      Generalize information from letures and readings. &	Develop a proposal
      that would $ \ldots $? \\
      Condense and re-state the content in one or two sentences. & How can you
      paraphrase this information into 1-2 concise sentences? \\
      Compare and contrast.	& What makes $ \ldots $ similar and different from $
      \ldots $? \\
      \hline
    \end{tabular}
    \caption{Level 5: Synthesize}
  \end{center}
\end{table}
%
\subsection{Level 6: Evaluate}
Evaluating means making judgments about something based on criteria and
standards. This requires checking and critiquing an argument or concept to form
an opinion about its value. Often there is not a clear or correct answer to this
type of question. Rather, it’s about making a judgment and supporting it with
reasons and evidence.
%
\begin{table}[H]
  \begin{center}
    \begin{tabular}[c]{p{6cm}p{6cm}}
      \hline
      Study methods & Types of questions to ask yourself \\
      \hline
      Decide if you like, dislike, agree, or disagree with an author or a
      decision. &	What is your opinion about $ \ldots $? What evidence and
      reasons support your opinion? \\
      Consider what you would do if asked to make a choice. &	How would you
      improve this? \\
      Determine which approach or argument is most effective. &	Which argument
      or approach is stronger? Why? \\
      \hline
    \end{tabular}
    \caption{Level 6: Evaluate}
  \end{center}
\end{table}
%
\subsection{Level 7: Create}
Creating involves putting elements together to form a coherent or functional
whole. Creating includes reorganizing elements into a new pattern or structure
through planning. This is the highest and most advanced level of Bloom’s
Taxonomy.
%
\begin{table}[H]
  \begin{center}
    \begin{tabular}[c]{p{6cm}p{6cm}}
      \hline
      Study methods & Types of questions to ask yourself \\
      \hline
      Build a model and use it to teach the information to others. & How can you
      create a model and use it to teach this information to others? \\
      Design an experiment. &	What experiment can you make to demonstrate or
      test this information? \\
      Write a short story about the concept. & How can this information be told
      in the form of a story or poem? \\
      \hline
    \end{tabular}
    \caption{Level 7: Create}
  \end{center}
\end{table}
%
\end{document}
