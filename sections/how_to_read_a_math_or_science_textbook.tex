\documentclass[../main.tex]{subfiles}
\graphicspath{{\subfix{../figures/}}}
%
\title{How To Read A Math(Science) Book}
%
\begin{document}
\maketitle
%
\section{How to read a math or science textbook}
\emph{NOTE: This passage was written with math textbooks in mind.
It is equally appropriate for a chemistry or physics textbook,
so please substitute science for math, depending on your class.}

Reading a mathematics text is very different from reading ordinary English.
Trying to read math the same way as a novel or a history text is certain to cause you
trouble. Math text typically alternates passages of explanation in English with pieces of
mathematics. Even its English, however, is of a special and stylized kind.

When reading any explanatory material in a math text,
the main principle is simple: \textbf{read every word, one word at a time}.
There is no ``catching the drift'' and then filling in from later paragraphs.
The authors will say his or her piece only once. If the author does seem to repeat,
it is probably to make a slightly different point.
\textbf{Every word counts, even a two-letter word}!

Forget about speed-reading; one small paragraph may take ten minutes!
Understand each part of each sentence as you go along. Keep in mind that familiar words
may have mathematical meanings that you need to learn. Of course, you need to know
the mathematical meanings of those unfamiliar mathematical words, too. If you cannot
understand the passage, then stop to think and go over mathematical definitions of words
you are reading until you figure out the meaning. Often it is necessary to start reading at
the beginning of the passage several times. Take your time. There is simply no way to
rush the process. It is a slow and delicate business…and ``\textbf{slow is fast!}''

With actual mathematics (equations and such), the trick is simply to see how each
line follows from the line before. If the step is especially obscure, the author may
provide some written explanation. But be sure that you understand where each line
comes from, before you go on to the next line. If you skip even one step, the rest of the
steps will make little or no sense to you.

It is best to \textbf{read mathematics with pencil and paper at hand}, and to reproduce it
yourself as you go along. But do not merely write down what you see in the book.
Instead, try to work out each line for yourself, step by step, with the author.

Really important mathematical passages are problems that the author has worked
out in detail. Successful students rely on these very heavily. One widely used series of
review texts in mathematics consists entirely of solved problems.

\emph{How to work a solved problem in the textbook}: \\
Start with a solved problem, working it through to the end, one step at a time.
Then close the book, and try to work it through again on your own. Do the whole
problem as many times as necessary, until you can reproduce the whole solution with the
book closed. But try not to memorize the solution. Instead, keep track of ``what to do'' to
move from each line to the next of the operations. Your version of the solution will
probably have more lines than the author's. That's good. It may sometimes take you two
or three steps to accomplish what the author does in one.

Working solved problems does take a lot of time. It is not unusual to spend an
hour or two on a single page. \textbf{Try to be patient}.

After you can work through the solved problems on your own, the homework
exercises will give you little trouble, for they are usually very similar. Exam questions,
too, will mostly follow the same pattern. In short:
\textbf{time spent on problems the author has solved for you will pay off in high grades.}

Some students are discouraged to see how easily the author sails through a tough
problem. ``I never would have thought to do that.'' ``How did s/he know that
adding $ x^2 $ to both sides would make it come out?'' ``If we're supposed to do this on an exam, I'm
finished.'' These are common reactions. But what you see in the book is only the
author's final product, carefully cleaned up for publication. He or she produced
wastebaskets full of scratch paper to find that clean solution. And teachers do the same
when preparing for a lecture. We math people make math look easy because we work
hard at it when you are not looking.

Remember that you will not be expected to invent a new problem-solving
technique on the exam. Your task is to do the techniques already shown to you in class
and in the book.

Some math texts use pictorial illustrations, and some do not. Like the words, the
picture needs slow and careful study. A quick glance will not do, as it might in a biology
text. Every line, every symbol is there for a specific reason, and you should take the time
to understand the picture thoroughly---in detail. This is especially true of graphs and
charts, which often contain a great deal of information in a small space.

\textbf{As you can see, you do not merely read a math test---you work through it!} The
information has to be dug out, not just skimmed from the surface. It is a slow business,
and the only good way to understand what the math text is trying to tell you.

\begin{center}
  \textbf{SO BE PATIENT.} \\
  \textbf{REMEMBER THAT ``SLOW IS FAST,''} \\
  \textbf{and} \\
  \textbf{ENJOY MATH READING!}
\end{center}
%
\end{document}
