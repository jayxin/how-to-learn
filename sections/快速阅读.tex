
\documentclass[../main.tex]{subfiles}
\graphicspath{{\subfix{../figures/}}}
%
\title{快速阅读}
%
\begin{document}
%
\maketitle
%
\begin{cuenotes}
  \cue{什么是快速阅读?}
  \note{
    快速阅读 (rapid reading) 起源于20世纪初,
    是一种让我们能够从文字材料中迅速接受信息的阅读法。
    快速阅读区别于传统的一字一句的慢阅读。
    它是充分调动左右脑功能,
    将被阅读的文字以组或行、
    块为单位进行大小不一的整体阅读。
  }
  \cue{快速阅读的原理}
  \note{
    \begin{itemize}
      \item \textbf{组块阅读,增加阅读视幅}:
        在阅读过程中杜绝一个字一个字的阅读习惯,
        一个视点多看几个字,从最简单的一个词组可以是一个组块,
        慢慢增加到半行可以是一个组块,一行是一个组块,
        两行是一个组块,甚至一页为一个组块。
      \item \textbf{减少默读频率}:
        默读是一种半意识状态,你现在看这行字的时候是不是也在默读呢?
        通过提速训练,会降低默读的频率,阅读的效率也会得到提升。
      \item \textbf{减少回看频率}:
        很多读者为了理解所读的内容,总是有意识地返回去重读。
        这种回看增加了每行的凝视次数,使阅读过程变慢。
    \end{itemize}
  }
  \cue{快速阅读的做法}
  \note{
    \textbf{眼动速读}: \\
    培养阅读能力的第一步,就是学会活动双眼。
    在阅读中,我们会下意识进行``默读''。虽然我们嘴唇并没有动,
    但是脑内掌管发声的区域仍然处于活跃状态,导致信息传送滞后。
    \begin{itemize}
      \item 默读下信息接收: 文字信息 $ \rightarrow $ 眼睛看到,嘴巴读出
        $ \rightarrow $ 声音信息 $ \rightarrow $ 耳朵听到,大脑接收
        $ \rightarrow $ 理解意愿.
      \item 视读下信息接收: 文字信息 $ \rightarrow $ 眼睛看到,大脑接收
        $ \rightarrow $ 理解意愿.
    \end{itemize}
    人的眼睛捕捉字符的最高极限大约是120万字/分钟。
    而人的嘴巴所能读出音节的最高速度是600音节/分钟。
    若不改掉声读的习惯,那么阅读的速度最快也就每分钟600字。 \\
    遗憾的是,我们是无法完全做到``消声阅读''的,
    所以最佳的提高阅读速度的技巧是优化眼球的运动方式,提高阅读速度。
  }
  \note{
    \textbf{用手指或笔尖辅助阅读}: \\
    阅读的效率取决于视线移动的效率,而引导恰恰能优化眼球的运动,
    帮助提高阅读速度,避免视线和思绪的胡乱游移。
    用工具辅助阅读,比如当指尖或笔尖接触到书页的时候,
    我们往往会产生一种亲近书本的感觉,对进度的感知也会更直接。 \\
    辅助阅读可以让眼睛不会毫无规律地在密密麻麻的文字之间跳动,
    同时还能毫不费力地找到下一行的开头,从而节省了换行所需的那1/3的时间。
    这样子在无形中加快你的阅读速度。
  }
  \note{
    \textbf{逐步拓展视线范围}: \\
    我们的眼睛在对焦时读入的信息会被投射到视网膜上。
    可是,我们的视网膜并不是对每个位置的信息都有相同的辨别能力。
    投射在视网膜中央的图像会尤其清晰,而两侧边缘区的图像则会比较模糊。
    因此,我们要做的就是逐渐把自己的视线范围运用到阅读上来,
    让自己学会在一次定焦的同时读入尽量多的字词。
  }
  \cue{速读与思维导图}
  \note{
    阅读是一种沟通过程,在这个过程有非常重要的两个转折点:
    \begin{enumerate}
      \item 作者要把自己的发散性思维转换为线性的表达方式,
        通过文字进行表达。
      \item 读者通过读取这些线性信息,在自己的大脑进行解码,
        生成一幅发散性图像,从而理解作者的意图。
    \end{enumerate}
    完成第二步最有效的方法就是利用思维导图来进行总结、
    梳理并检查自己的思维逻辑。 \\
    在制作书的思维导图的时候,我们可以模拟作者构思这本书的过程,
    按照从大到小、从整体到部分、从概况到细节的模式去提取关键词。
    制作的过程大致包括以下三个步骤:
    \begin{enumerate}
      \item 从封面和前言先整体地了解这本书的核心观点是什么。
      \item 在阅读目录的时候提取第一层关键词,了解作者主要讲述哪几个方面的问题。
      \item 结合眼动速读和扩大视线范围的速读方法,从行文中提取关键词,
        详细了解每个问题分别讲了哪些内容。
    \end{enumerate}
  }
\end{cuenotes}
%
\end{document}
