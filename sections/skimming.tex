\documentclass[../main.tex]{subfiles}
\graphicspath{{\subfix{../figures/}}}
%
\title{Skimming}
%
\begin{document}
\maketitle
%
\section{Introduction}
Do you ever feel like you spend way too much time reading? Do you have trouble
determining which parts of a text are the most important? Do you wish you could
collect information from books in a quicker and more efficient way? If so,
consider skimming the next time you sit down to read a text.
%
\section{What is skimming?}
Skimming is a strategic, selective reading method in which you focus on the main
ideas of a text. When skimming, deliberately skip text that provides details,
stories, data, or other elaboration. Instead of closely reading every word,
focus on the introduction, chapter summaries, first and last sentences of
paragraphs, bold words, and text features. Skimming is extracting the essence of
the author’s main messages rather than the finer points.
%
\section{Why skim?}
\begin{cuenotes}
  \cue{You need the “big picture” or main points when you’re reading.}
  \note{
    Even if you’re going to do a more detailed reading of the text, skimming as
    a form of previewing can help you better comprehend what you read. Knowing
    when and how to skim will help you become a more efficient, strategic
    reader. You’ll become better at determining what parts of the text are most
    important. There may also be times when your professor wants you to
    understand the big picture, not all of the little details. In these cases,
    skimming helps you understand the overall points of the text and its
    relevance to your course without bogging you down.
  }
  \cue{Make the most of your time.}
  \note{
    Sometimes you don’t have time to do everything. With skimming, you’ll be
    able to cover vast amounts of material more quickly and save time for
    everything else that you have on your plate. Maybe you don’t have time to
    finish your reading before class, but skimming will help you get the main
    points and attend class much more prepared to maximize in-class learning.
  }
  \cue{You need to review a text you have read before.}
  \note{
    Skimming is also an efficient way to refresh your memory of large amounts of
    material before an exam. Skimming a text that you have already read helps
    you recall content and structure.
  }
\end{cuenotes}
%
\section{Skimming is not $ \ldots $}
Skimming can present problems if not done intentionally. Skimming is not simply
flipping through a text quickly or paying half attention to it. When skimming,
be deliberate and intentional with what you choose to read, and make sure that
you are focused. Skimming is not a lazy way out or a half-hearted attempt at
reading. Make sure that you use it carefully and strategically and are able to
walk away with the main ideas of the text.
%
\section{Skimming methods}
\begin{cuenotes}
  \cue{Beginnings \& endings}
  \note{
    Read first and last sentences of paragraphs, first and last paragraphs of
    major sections, and introductions and summaries of chapters.
  }
  \cue{Wheat vs. chaff}
  \note{
    Read only the amount of text necessary to determine if a section presents a
    main idea or support for a main idea.
  }
  \cue{Visual \& verbal cues}
  \note{
    Watch for signal words and phrases that indicate an author’s direction
    (e.g., however, although, moreover, in addition to). Things to focus on
    while skimming:
    \begin{itemize}
      \item Introduction and conclusion
      \item Chapter/section summaries
      \item First and last sentences
      \item Titles, subtitles, and headings
      \item Bold words
      \item Charts, graphs, or pictures
      \item End of chapter review questions
    \end{itemize}
  }
\end{cuenotes}
%
\section{When to skim}
There are certain texts that lend themselves to skimming better than others. It
is typically less beneficial to skim novels, poetry, and short stories or texts
that do not have text features such as such as tables of content, chapter or
section summaries, headings, bold words, pictures, and diagrams. Non-fiction
texts, like textbooks, journal articles, and essays are typically full of these
kinds of text features and are more suited for skimming.

Skimming can also be a good tool for conducting research and writing papers.
Typically, when researching or writing, you will not need to read every word of
every text closely, but will benefit more from skimming while evaluating your
sources or identifying information important to your work.

Finally, know your context. There may be some texts that you are better off
reading closely and thoroughly. Some professors specifically tell you that they
include small details from the textbook on exams. You may have some classes that
are just difficult to understand, and you may find that reading closely helps
you comprehend concepts better. Before skimming, spend some time thinking about
your classes, professors, and needs to determine if you have any texts you may
need to read more closely.
%
\section{Active reading strategies}
When skimming, it’s important to continue to use active reading strategies. This
keeps your brain active, engaged, and focused, and helps you understand and
retain information better and longer. Here are a few effective active reading
strategies to pair with skimming:
%
\begin{cuenotes}
  \cue{Set a purpose for reading.}
  \note{
    Instead of approaching the text as something you just have to get through,
    identify a purpose for this reading. What do you want to get out of it? Why
    are you reading it? Keep this purpose in mind as you read.
  }
  \cue{Preview.}
  \note{
    Look through the text before started to read and focus on headings,
    illustrations, captions, highlighted items, end of chapter summaries, etc.
    These features give you an idea of the main concepts of the text and what
    you should focus on while skimming.
  }
  \cue{Make a prediction.}
  \note{
    Right after previewing, make a prediction about what you think the chapter
    or section is going to be about.
  }
  \cue{Activate prior knowledge.}
  \note{
    Make a list of what you already know about the topic and what you want to
    know about it. Identify and write down any questions you have.
  }
  \cue{Annotate.}
  \note{
    Instead of copying down copious notes from the book, jot down brief notes
    and thoughts (in your own words) in the margins of the text or in a
    notebook. PDF viewers, such as Adobe Acrobat and Preview, also let you add
    notes directly on the page of a digital text. Other possibilities include
    note-taking apps such as Evernote, OneNote, or Google Keep.
  }
  \cue{Summarize the main ideas.}
  \note{
    After a section or page, stop and write a 1-3 sentence summary in your own
    words. This keeps your brain engaged and ensures you are comprehending what
    you read.
  }
  \cue{Generate questions.}
  \note{
    Ask and write down questions that you have as you read the text and/or
    questions that you would ask a class if you were the instructor. Try using
    different levels of questions.
  }
\end{cuenotes}
%
\end{document}
