\documentclass[../main.tex]{subfiles}
\graphicspath{{\subfix{../figures/}}}
%
\title{Sleeping To Succeed}
%
\begin{document}
\maketitle
%
\section{Introduction}
Ernest Hemingway is said to have once remarked, ``I love sleep. My life has a
tendency to fall apart when I'm awake.'' Whether you have it all together during
the day or feel more like Hemingway, we all benefit from healthy sleep habits.
Sleep promotes cognition and memory, facilitates learning, recharges our mental
and physical batteries, and generally helps us make the most out of our days.
With plentiful sleep, we improve our mental and physical health, reduce stress,
and maintain the routine that is critical to healthy daily functioning.

Within the busy schedules of college students, sleep is often the first thing to
go when trying to squeeze in all of the academic, social, and extracurricular
activities that are often part of campus life. And when you're taking online
classes remotely, you may find yourself catching up on asynchronous course
content at any hour of day or night while the rest of the household sleeps. This
handout discusses why it is important to maintain healthy sleep habits and
provides tips and tricks on how to do it!
%
\section{Why is sleep so important?}
Sleep plays a critical role in helping our bodies and minds recover and
rejuvenate. As a result, sleep contributes to improvements in learning and
promotes regulatory functions such as emotional and behavioral control that are
important for each and every day. Some examples of physiological and behavioral
benefits of sleep include:
%
\begin{itemize}
  \item Improving our ability to learn new information and form memories
  \item Restoring neural connections
  \item Assisting in optimal emotional control, decision making, and social interaction
\end{itemize}
%
\section{How much sleep do you need?}
The optimal amount of sleep for each person may vary, but generally research
suggests 7-9 hours per night for college-aged populations.
%
\section{How much sleep are college students getting?}
As you might guess, most college students do not get the recommended amount of
sleep necessary to maximize its benefits. Sleep is particularly important for
college students because sufficient sleep has been linked to increases in GPA!
Research has found:
%
\begin{itemize}
  \item 50\% of college students report daytime sleepiness, and 70\% report
    insufficient sleep.
  \item The GPAS of students receiving 9+ hours of sleep per night were
    significantly higher (3.24) than those of students receiving 6 or fewer
    hours of sleep per night (2.74).
\end{itemize}
%
\section{What if you're not getting enough sleep?}
Because sleep plays such a crucial role in human functioning, lack of sleep can
lead to a number of consequences affecting behavior, memory, emotions, and
learning when we are awake. These consequences can include:
%
\begin{itemize}
  \item Inattention, irritability, hyperactivity, poor impulse control and
    difficulty multi-tasking
  \item Impaired memory
  \item Impaired math calculation skills
\end{itemize}
%
In extreme sleep deprivation, consequences can even include mood swings and
hallucinations.

When we do not get the sleep we need, our bodies do not forget; we go into sleep
debt. Our bodies continue to pay back this debt by trying to get sleep whenever
possible, which can result in microsleeps.

You may not notice inadvertent sleeping during the day (even in class or when
studying!) that can last just seconds. These microsleeps impede concentration
and negatively impact retention of information.

Additionally, individuals often use caffeine or others stimulants to stay awake.
This not only puts them at risk for the consequences of poor sleep, but also the
negative health effects of increased stimulant consumption.
%
\section{What types of things affect falling and staying asleep?}
Sleep can be affected by a number of things including how we treat our bodies,
what we put in our bodies, and how we interact with our environment:
%
\begin{itemize}
  \item Caffeine
  \item Screen light
  \item Sleep routines (regular bedtime)
  \item Exercise
  \item Diet
  \item Decongestant stimulants and/or diet pills
  \item Nicotine
  \item Alcohol
\end{itemize}
%
Although alcohol may help you fall asleep because it is a depressant, it reduces
sleep stages II, IV, and REM, which are the restorative sleep stages.
%
\section{How to optimize your sleep}
Given what we know about sleep, there are a number of things you can do and
avoid to improve your sleep cycle. This list is not exhaustive, but it includes
many suggestions that help in falling and staying asleep so you can get the 7-9
hours your body and mind need.

THINGS YOU MAY WANT TO TRY
%
\begin{itemize}
  \item Allow yourself enough time to sleep.
  \item Gradually set earlier bedtimes when attempting to adjust your sleep
    cycle.
  \item Expose yourself to bright light in the morning to help wake up.
  \item Keep your bedroom cool, dark, and quiet to help fall asleep.
  \item Exercise regularly but not right before bed.
  \item Maintain a regular sleep routine on weekdays and weekends.
  \item Relax yourself as much as possible before bed. This can include taking a
    warm bath, meditating, or reading something that is not cognitively taxing.
  \item Re-evaluate your daily schedule and make time for 7-9 hours of sleep
    every night.
  \item Prioritize and protect your sleep time. Find a friend who can help keep
    you accountable for going to sleep at your goal bedtime each night.
  \item Structure your day and plan ahead on your exams, assignments, due dates,
    and activities so that you don't have to end up cramming or working late
    into the night. Use a weekly calendar and/or a priorities list to help take
    control of your to do lists and better manage your time to prioritize sleep.
\end{itemize}
%
THINGS YOU MIGHT WANT TO CUT OUT
%
\begin{itemize}
  \item Don't use alcohol to help fall asleep. While this may help fall asleep,
    you may be more likely to have difficulty staying asleep as alcohol can
    disrupt the natural cycle of sleep, and the sleep may be less restorative.
  \item Don't eat large meals right before bed.
  \item Don't engage in rigorous exercise before bed.
  \item Don't use nicotine. Nicotine is a stimulant, and daytime use can inhibit
    sleep.
  \item Don't drink caffeine within 8hrs of your intended bedtime.
  \item Don't expose yourself to bright lights before going to bed.
  \item Don't use electronic devices that give off light such as TV, computer,
    phones, etc. before bed. This light inhibits the secretion of melatonin
    making it more difficult to fall asleep.
\end{itemize}
%
\end{document}
