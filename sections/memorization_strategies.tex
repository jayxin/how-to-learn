\documentclass[../main.tex]{subfiles}
\graphicspath{{\subfix{../figures/}}}
%
\title{Memorization Strategies}
%
\begin{document}
\maketitle
%
\section{Introduction}
Many college courses require you to memorize mass amounts of information.
Memorizing for one class can be difficult, but it can be even more frustrating
when you have multiple classes. Many students feel like they simply do not have
strong memory skills. Fortunately, though, memorizing is not just for an elite
group of people born with the right skills---anyone can train and develop their
memorizing abilities.

Competitive memorizers claim that practicing visualization techniques and using
memory tricks enable them to remember large chunks of information quickly.
Research shows that students who use memory tricks perform better than those who
do not. Memory tricks help you expand your working memory and access long term
memory. These techniques can also enable you to remember some concepts for years
or even for life. Finally, memory tricks like these lead to understanding and
higher order thinking. Keep reading for an introduction to effective
memorization techniques that will help you in school.
%
\section{Simple memory tips and tricks}
In addition to visual and spatial memory techniques, there are many others
tricks you can use to help your brain remember information. Here are some simple
tips to try.
\\

\begin{cuenotes}
  \cue{Try to understand the information first.}
  \note{
    Information that is organized and makes sense to you is easier to memorize.
    If you find that you don’t understand the material, spend some time on
    understanding it before trying to memorize it.
  }
  \cue{Link it.}
  \note{
    Connect the information you are trying to memorize to something that you
    already know. Material in isolation is more difficult to remember than
    material that is connected to other concepts. If you cannot think of a way
    to connect the information to something you already know, make up a crazy
    connection. For example, say you are trying to memorize the fact that water
    at sea level boils at 212 degrees Fahrenheit, and 212 happens to be the
    first three digits of your best friend’s phone number. Link these two by
    imagining throwing your phone into a boiling ocean. It’s a crazy link, but
    it can help that fact to stick.
  }
  \cue{Sleep on it.}
  \note{
    Studies show that your brain processes and stores information while you
    sleep. Try to review information just before you go to sleep---even if it’s
    only for a few minutes---and see if it helps embed the information in your
    memory.
  }
  \cue{Self-test.}
  \note{
    Quiz yourself every so often by actively recalling the information you are
    trying to study. Make sure to actively quiz yourself---do not simply reread
    notes or a textbook. Often, students think they remember material just
    because it is familiar to them when they reread it. Instead, ask yourself
    questions and force yourself to remember it without looking at the answer or
    material. This will enable you to identify areas that you are struggling
    with; you can then go back to one of the memory tricks to help yourself
    memorize it. Also, avoid quizzing yourself immediately after trying to
    memorize something. Wait a few hours, or even a day or two, to see if it has
    really stuck in your memory.
  }
  \cue{Use distributed practice.}
  \note{
    For a concept to move from your temporary working memory to your long-term
    memory, two things need to happen: the concept should be memorable and it
    should be repeated. Use repetition to firmly lodge information in your
    memory. Repetition techniques can involve things like flash cards, using the
    simple tips in this section, and self-testing. Space out your studying and
    repetition over several days, and start to increase the time in between each
    study session. Spacing it out and gradually extending the times in between
    can help us become more certain of mastery and lock the concepts into place.
  }
  \cue{Write it out.}
  \note{
    Writing appears to help us more deeply encode information that we’re trying
    to learn because there is a direct connection between our hand and our
    brain. Try writing your notes by hand during a lecture or rewriting and
    reorganizing notes or information by hand after a lecture. While you are
    writing out a concept you want to remember, try to say the information out
    loud and visualize the concept as well.
  }
  \cue{Create meaningful groups.}
  \note{
    A good strategy for memorizing is to create meaningful groups that simplify
    the material. For example, let’s say you wanted to remember the names of
    four plants---garlic, rose, hawthorn, and mustard. The first letters
    abbreviate to GRHM, so you can connect that with the image of a GRAHAM
    cracker. Now all you need to do is remember to picture a graham cracker, and
    the names of the plants will be easier to recall.
  }
  \cue{Use mnemonics.}
  \note{
    Mnemonics are systems and tricks that make information for memorable. One
    common type is when the first letter of each word in a sentence is also the
    first letter of each word in a list that needs to be memorized. For example,
    many children learned the order of operations in math by using the sentence
    Please Excuse My Dear Aunt Sally (parentheses, exponents, multiply, divide,
    add, subtract).
  }
  \cue{Talk to yourself.}
  \note{
    It may seem strange at first, but talking to yourself about the material you
    are trying to memorize can be an effective memory tool. Try speaking aloud
    instead of simply highlighting or rereading information.
  }
  \cue{Exercise!}
  \note{
    Seriously! Studies show that exercise can improve our memory and learning
    capabilities because it helps create neurons in areas that relate to memory.
    Cardio and resistance training (weights) both have powerful effects, so do
    what works best for you.
  }
  \cue{Practice interleaving.}
  \note{
    Interleaving is the idea of mixing or alternating skills or concepts that
    you want to memorize. For example, spend some time memorizing vocabulary
    words for your science class and then immediately switch to studying
    historical dates and names for your history class. Follow that up with
    practicing a few math problems, and then jump back to the science
    definitions. This method may seem confusing at first, but yields better
    results in the end than simply spending long periods of time on the same
    concept.
  }
\end{cuenotes}
%
\section{Visual and spatial techniques}
Visual and spatial techniques are memory tricks that involve your five senses.
They utilize images, songs, feelings, and our bodies to help information stick.
Humans have outstanding visual and spatial memory systems. When you use visual
and spatial memory techniques, you use fun, memorable, and creative approaches
rather than boring, rote memorization. This makes it easier to see, feel, or
hear the things you want to remember. Visual and spatial techniques also free up
your working memory. When you group things together, you enhance your long-term
memory. Using visual and spatial techniques helps your mind focus and pay
attention when your mind would rather wander to something else. They help you
make what you learn meaningful, memorable, and fun.

The common practice of using your knuckles to remember the number of days in
each month is a great example of an easy visual spatial technique to help you
remember details.
\\

\begin{cuenotes}
  \cue{Memorable visual images.}
  \note{
    The next time you have a key item you need to remember, try making a
    memorable visual image to represent that item. Images are important because
    they connect directly to your brain’s visuospatial centers. Images help you
    remember difficult concepts by tapping into visual areas. But you don’t just
    have to use images---the more of the five senses you can use, the easier it
    will be for you to recall information. Rather than just visualizing an
    image, try to smell, feel, and hear the image as well. For example, if you
    are trying to remember that the capital of Louisiana is Baton Rouge, draw up
    an image of a girl named Louise carrying a red baton.
  }
  \cue{The memory palace technique.}
  \note{
    This technique involves visualizing a familiar place---like the layout of
    your house or dorm room---and using it as a visual space where you can
    deposit concept-images that you want to remember. This technique can help
    with remembering unrelated items, like a grocery list. To use the memory
    palace technique, visualize your place (house or dorm room) and then imagine
    items from your grocery list in different areas around the place. For
    example, picture a cracked egg dripping off the edge of the table or a
    bushel of apples sitting on the couch. This technique can take some time to
    get used to, but once you do, the quicker and more effective it becomes.
  }
  \cue{Songs and jingles.}
  \note{
    Much like the memory palace and images, songs or jingles use your brain’s
    right hemisphere and can help us remember tricky things like equations and
    lists. There are already plenty of songs out there for things like the
    quadratic formula---try Googling what you are trying to remember to see if
    someone has already created a tune. If not, try making your own.
  }
  \cue{The five senses.}
  \note{
    Using as many of the five senses as possible when studying helps you use
    more parts of your brain and retain information better. For example, if
    studying for an anatomy exam, pick up the anatomy models, feel each part,
    and say the names of them out loud.
  }
  \cue{Lively visual metaphors or analogies.}
  \note{
    This can help you to not only remember but understand concepts, especially
    in math and science. A metaphor is a way of realizing that one thing is
    somehow similar to another. For example, think about the country of Syria as
    shaped like a bowl of cereal and the country Jordan as a Nike Air Jordan
    sneaker. Metaphors---especially visual ones---can stick with you for years.
    They help glue ideas in your mind because they make connections to neural
    structures that are already there.
  }
\end{cuenotes}
%
\section{Final thoughts}
Some of these techniques can feel strange at first or take some time to develop.
The more you practice them, the easier and more natural they become, and the
more information you can commit to memory. Also, remember that you do not need
to do every tip on this list. Experiment with a few and find which ones work for
you.
%
\end{document}
