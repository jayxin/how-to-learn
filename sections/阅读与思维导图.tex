\documentclass[../main.tex]{subfiles}
\graphicspath{{\subfix{../figures/}}}
%
\title{阅读与思维导图}
%
\begin{document}
%
\maketitle
%
\begin{cuenotes}
  \cue{为什么用思维导图记读书笔记?}
  \note{
    \begin{itemize}
      \item 把一本书变成一张纸,加深理解和记忆,方便复习
      \item 提炼关键词信息,辅助阅读和思考
      \item 让你的思维逻辑更清晰,建立自己的知识网
      \item 书看得多了,学习和理解到的知识点就越多,
        这个时候你就要化零为整,去理清每个知识点之间的关联,
        用思维导图做出来的笔记,逻辑清晰,可以帮助你把碎片、
        零乱的笔记从复杂变得有序,梳理和建立自己的知识网络。
    \end{itemize}
    在日常阅读中,如果能够在阅读过程中将输入和输出有效地结合在一起,
    利用线性关系厘清书本行文逻辑和作者的感情逻辑,更能提升学习效率。
  }
  \note{
    在《思维导图阅读法》中,作者将阅读理解力分为四个层次:
    \begin{enumerate}
      \item 掌握关键点,5W2H(人、事、时、地、物、因、果、成本)。
      \item 掌握重点间彼此的逻辑关系。
      \item 了解隐含的意义以及其与自己的关系。
      \item 思考如何将理论实践于自身。
    \end{enumerate}
    可以看到,阅读的终极目标,就是内化作者的观点,
    并将作者的观点与自身经验结合,将策略自由运用在自己身上,
    对人生有所帮助。这四个层次是循序渐进的。
  }
  \note{
    并不是所有的书都适合用思维导图做读书笔记,
    像杂文、散文、随笔之类的文章,一般没有明显的结构。
    思维导图这种笔记方式更适合用于构建知识体系,
    因为在绘制思维导图的过程中你可以:
    \begin{itemize}
      \item 梳理—构建知识体系
      \item 关联—和已有的知识体系产生联系
      \item 回顾—加深对知识的吸收
      \item 加工—联系实际进行输出
    \end{itemize}
    通过思维导图,不断层次化你的内容,让知识点之间的联系更紧密,
    从而做到心中有网,融会贯通。
    而且,这种笔记方式是可以无限延展的,每隔一段时间,
    可以在原来的导图上进行新内容的添加,有什么新的感想和体会也可以加上去。
    此外,自我检测和与他人分享是检验你对知识的理解程度很好的方式。
    比如说你可以拿着这张图问自己``这本书主要讲了哪些内容'',
    也可以和别人解释和分享。
    这样一来,你就要用自己的语言去串联和描述书中的知识,
    不知不觉中就用到了``费曼技巧''。
  }
  \cue{如何用思维导图写读书笔记?}
  \note{
    \begin{enumerate}
      \item 对照书本目录,按照作者的章节排序做第一张思维导图的目录笔记.
        有些朋友看完一本书不知道要如何去理解它,那么你直接对着书本的章节目录去做一张思维导图,在这一步不需要你加上自己的理解去进行修改,但如果标题太长,你也可以提取关键词去精简文字。
      \item 仔细看看这本书的封面、封底、腰封以及目录、还有前言的内容,
        大致了解这本书后,可以做第二张思维导图了。 \\
        主要了解这三个点:
        \begin{itemize}
          \item 这本书主要讲的是什么?
          \item 作者为什么要写这本书?
          \item 这本书能解决什么问题?
        \end{itemize}
        若你能针对了解到的信息,在读这本书之前就提出三个问题,
        就更好了。 \\
        因为二八定律存在于任何领域,书也是这样的。
        一本书不可能全部都是精华要点,如果你能找到其中的20\%
        进行精读,其他80\%进行泛读、速读,
        那么你肯定会又快又好的读完一本书。
        而能不能找到那其中的20\%,就看你有没有在阅读前做自我提问了,
        带着问题去读书,会让你更有效率;
        这就像你去商场买鞋,如果你一开始就知道你想买的是一双什么鞋,
        那么你一到商场就会直奔主题,而不会东逛西看,浪费时间。
      \item 开始进入正文的阅读部分,进行标注.
        看到有感触、想法的内容、可以用笔划线、重要的内容可以折个角,
        比如重要的概念解释、名人金句、方法、案例,
        作者观点等这些内容都是我们在看书的过程中可以去标注的。
        有人会建议我们在看书的时候,边看边记笔记,
        但是我自己看书的习惯是,中途看书不要停,我会在看完之后统一去做好笔记。
        正文部分的内容,可以结合你的阅读习惯,精读、快读都可以。
        我是精读和快读结合来的,看到我感兴趣的,对我来说比较重要的内容,
        我才会精读,对于一些看不太懂、不理解,对我目前作用不大的,
        我会快读、甚至是跳读。
        不要担心,快读或跳读会不会影响我们的阅读效果,
        你要学着相信自己的大脑,放松身体,享受看书的过程,
        而不是担心我会错过什么.
        如果你在开始阅读前就做好准备,
        潜意识会`引'着你读到重点部分,所以我觉得用什么方法来读书,
        并不重要,重要的是,它对你有用就行。
        就像俗话说的:``别管黑猫白猫,能抓到老鼠的就是好猫'',
        我看书就经常这样,别管别人说什么,只要我看到了,觉得不错,
        我就会去试一下,好用就借鉴,不好用就不用。
      \item 读完后、填充和整理第二张思维导图的笔记.
        回看第一张和第二张导图,看看你提出的问题,在哪个章节部分有答案。
        这个时候,你可以直接用你的问题作为导图的一个分支节点,
        去总结归纳知识点,进行列点梳理;
        比如要解决这个问题,你要知道这个问题的概念、导致的原因、
        以及解决的方案分别是什么,
        你能不能通过关键词用自己的话去概述出来?
        如果不能,也不用急,去针对需要总结的知识点可以回看第二遍,
        还是用关键词去提炼,然后填入你的思维导图。
        要做好这一步,其实还是有一定难度的,因为它会比较考验你的耐心,
        能力还有执行力的问题。
        如果你的耐心不够,能力也有欠缺,执行力又不强,
        那么这一步你很可能就进行不下去了。
        但是我觉得,如果你是想真正通过读书去改变自己,
        实现自我成长和升级,那么你要去学会输出,做好读书笔记是必不可少的一个环节。
        从输入到输出,再到输入,循环往复地不停歇,
        如果你能跟着这个学习的循环结构去行动,那么在这个过程当中,
        你就会慢慢的进步和成熟。
        书会越看越多,越看越快,笔记会写得又快又好,你思考问题的角度,
        观点也会越来越深刻。读书写作不分家,
        它们不会让你很快就获得改变,但却会细水长流,
        慢慢滋养你的内在和灵魂。
    \end{enumerate}
    }
    \note{
      \begin{itemize}
        \item 根据书的目录,大致了解整本书的内容,
          确定思维导图的基本结构。
        \item 进行泛读,挑出哪些部分值得精读,
          并大概了解作者整本书想讲什么。
        \item 开始进行精读,在精读的时候,画出关键词,
          在思维导图上进行细节的填充。
        \item 填充完成后,进行进一步的删减和整理,
          最后完成整本书思维导图的绘制。
      \end{itemize}
    }
    \cue{用思维导图绘制读书笔记的要点}
    \note{
      \textbf{提取关键词}: \\
      相比于大纲笔记,用思维导图制作读书笔记难度在于关键词的提取。
      在绘制思维导图的过程中,建议避免使用长句。
      避免大段文字堆砌能极大地提升整张导图的可读性,
      而且提取关键词的过程是压缩信息的过程,
      也是一个变被动吸收为主动思考的过程。\\
      在进行关键词提取时可以遵循以下原则:
      \begin{itemize}
        \item 运用能阐明关键概念的词,名词为主、动词次之,
          辅之以必要的形容词和副词。
        \item 精简到不能精简为止。
      \end{itemize}
    }
\end{cuenotes}
%\summary{}
%
\end{document}
