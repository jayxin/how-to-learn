\documentclass[../main.tex]{subfiles}
\graphicspath{{\subfix{../figures/}}}
%
\title{高效搜索}
%
\begin{document}
\maketitle
%
\section{引子}
随着科技的发展,我们所能触及的内容和信息前所未有的广,
但获取高质量信息的成本却越来越高。不管是推荐算法还是竞价广告,
找到相关的高质量信息的过程都是困难重重。

如何更快更好地找到我们想要的信息?
和大家分享一些基本的搜索技巧以及判断所获取信息质量的方法。
%
\section{高效搜索的技巧}
说起找东西,很多人的第一直觉是搜索引擎。
正确运用搜索引擎是我们现代数字游民的基本素养。在工作或生活中,
不随便问搜索引擎能简单搜到的问题是一个有效藏拙的人际交往小技巧。
那么如何高效使用搜索引擎呢?和大家分享一些简单的技巧。
\\

\begin{cuenotes}
    \cue{找准关键词}
    \note{
      搜索引擎相当于一个巨大的信息库索引,
      而找准搜索的关键词就是快速找到想要的信息的关键。
      在确定搜索关键词时,可以注意以下几点:
      \begin{itemize}
        \item 尽量少用口语化的表达,尽量用精简的词来搜索;
        \item 换种语言可能搜到更高质量的内容,
          中文找不到时可以尝试用英文来搜索;
        \item 寻找案例或实践时可以加上tutorial、case、example、
          tricks等关键词。
      \end{itemize}
    }
    \cue{精准搜索}
    \note{
      明确搜索的关键词后,可以运用相应的搜索技巧来精准搜索你想获取的信息.
    }
    \cue{\texttt{"XX"}}
    \note{
      精准匹配, 搜索与双引号内文本完全相同的内容.
      e.g. \texttt{"庸俗与无聊"}
    }
    \cue{\texttt{A +B}}
    \note{
      搜索同时包含关键词 A 和 B 的内容,
      \texttt{+}前有空格.
      e.g. \texttt{胡兰成 +张爱玲}
    }
    \cue{\texttt{A -B}}
    \note{
      排除搜索, 搜索包含 A 但不包含 B 的内容,
      \texttt{-}前有空格.
      e.g. \texttt{胡兰成 -张爱玲}
    }
    \cue{\texttt{site:网址}}
    \note{
      站内搜索相关内容.
      e.g. \texttt{site:zhihu.com SEO}
    }
    \cue{\texttt{intitle:XX}}
    \note{
      搜索标题中包含关键词的结果.
      e.g. \texttt{intitle:kevin systrom}
    }
    \cue{\texttt{inurl:XX}}
    \note{
      搜索 URL 中包含关键词的结果.
      e.g. \texttt{inurl:metaverse}
    }
    \cue{\texttt{filetype:pdf}}
    \note{
      搜索特定格式的文件.
      e.g. \texttt{filetype:pdf digital minimalism}
    }
    \cue{垂直搜索}
    \note{
      当你在常见的搜索引擎内搜索不到想找的内容时,
      在垂类平台中搜或许会有新发现。很多平台的内容是无法被搜索引擎抓取的,
      比如微信生态内的内容。灵活使用微信``搜一搜''、小红书``搜一搜''、抖音``搜一搜''等方式,
      或许可以找到你想找的内容。
    }
\end{cuenotes}
%
\section{确认信息的可靠性}
掌握了基本的搜索技能后,更重要的一步是确认所得信息的可靠性。
人的很多决定都是基于收集到的信息做出的,因而在搜索资料的过程中,
应该有意识地去判断信息的可信度,我们可以借助 CARS 清单的这四个维度来进行判断。
\\

\begin{cuenotes}
  \cue{可靠性(Credibility)}
  \note{
    \begin{itemize}
      \item 来源是否可信和可靠
      \item 分清观点和信息
    \end{itemize}
  }
  \note{
    可靠性可以从以下几个维度来考量:
    作者的资质、相关的质量审查、网站/平台本身的权威性以及一些排版、
    语病等细节。拥有相关学术背景或相关从业经历的作者可信度会更高些,
    有严格质量审核和控制的信息和平台,
    比如学术刊物或出版物的可信度会更高些。相反,匿名、
    糟糕的视觉呈现和较多语病的信息,其可信度都要抱有警惕。
  }
  \cue{准确性(Accuracy)}
  \note{
    \begin{itemize}
      \item 信息是否正确
      \item 真实、全面、详细和时效
    \end{itemize}
  }
  \note{
    信息的准确性体现在时效性、细节和精准的维度上。
    在高速发展的领域,比如医药、科学和技术,信息的时效性尤其重要。
    高质量的信息在关键数据和事实上都有精准呈现,不会模糊化处理。
    此外,正反维度都有所呈现,
    没有故意忽略关键信息也是确认信息准确性的重要考量方式。
  }
  \cue{合理性(Reasonableness)}
  \note{
    \begin{itemize}
      \item 材料是否公平、客观、一致
      \item 是理性还是情绪
    \end{itemize}
  }
  \note{
    信息的合理性体现在信息的公正、客观中立和前后一致性上。
    在辨别信息时,注意信息的表达方式,
    沉静理性的说理方式会比煽动情绪更具合理性和说服力。
    保持完全客观很难,有些作者和机构有自己的天然立场。
    因而在摄取信息需时时保持理智,对言语过激的、浮夸的、
    情绪感染力强的文本保持警惕。
  }
  \cue{支持性(Support)}
  \note{
    \begin{itemize}
      \item 是否有其他验证
      \item 是否有充分的支持(数据/文献)
    \end{itemize}
  }
  \note{
    辨别信息是否高质量,很重要的一点是信息是否有理有据,
    能得到充足的支持和验证。在阅读时,
    我们需要自问下作者引用的信息来自哪里,是否有数据支持,
    是否有正确的索引(数据类相关的信息更需要清晰地索引其来处),
    还有同样的信息是否有其他资料的验证和支持,
    是否可利用信息间的交叉验证来确认。
  }
\end{cuenotes}

有意识地将这个框架应用在所接触的各种的信息中,
我们对所接触的内容会有更高的辨识度,
也更少会受到不实信息和偏见立场的影响。
%
\section{用 XMind 建立知识框架}
《认知天性》一书中将学习分为三个关键步骤:
%
\begin{itemize}
  \item 编码:把感觉感知到的东西转化为有意义的心理表征
  \item 巩固:把大脑中的新表征强化为长期记忆,识别并稳定记忆痕迹
  \item 检索:把学到的和不同种类的线索联起来
\end{itemize}
%
在获取信息的过程中,
有意识地用 XMind 对所获信息进行整理和概括即是编码的过程。
用自己的方式对内容进行总结和归纳,不仅能加强你对资料的理解,
更让你在内心形成有意义的心理表征,为之后进一步的巩固和检索打下基础。
比如之前看到咖啡的相关介绍,用 XMind 梳理后即对咖啡本身有了更清晰的认识。

经过处理后的信息更容易消化和理解,之后索引相关知识也更方便。
另外,XMind 软件内也支持搜索功能,当你做了一张内容繁杂的图时,
可以灵活应用搜索功能来快速定位内容。组合快捷键 Command+F/Ctrl+F 即
可唤出搜索的导航面板。

不管是做什么决策,不管是投资还是生活里的细碎小事,
学会 Do Your Own Research 都是每个人应该必备的技能。
希望以上的分享能有所帮助.
%
\end{document}
