\documentclass[../main.tex]{subfiles}
\graphicspath{{\subfix{../figures/}}}
%
\title{Study Smarter Not Harder}
%
\begin{document}
\maketitle
%
\section{Introduction}
Do you ever feel like your study habits simply aren't cutting it? Do you wonder
what you could be doing to perform better in class and on exams? Many students
realize that their high school study habits aren't very effective in college.
This is understandable, as college is quite different from high school. The
professors are less personally involved, classes are bigger, exams are worth
more, reading is more intense, and classes are much more rigorous. That doesn't
mean there's anything wrong with you; it just means you need to learn some more
effective study skills. Fortunately, there are many active, effective study
strategies that are shown to be effective in college classes.

This handout offers several tips on effective studying. Implementing these tips
into your regular study routine will help you to efficiently and effectively
learn course material. Experiment with them and find some that work for you.
%
\section{Reading is not studying}
Simply reading and re-reading texts or notes is not actively engaging in the
material. It is simply re-reading your notes. Only ‘doing' the readings for
class is not studying. It is simply doing the reading for class. Re-reading
leads to quick forgetting.

Think of reading as an important part of pre-studying, but learning information
requires actively engaging in the material (Edwards, 2014). Active engagement is
the process of constructing meaning from text that involves making connections
to lectures, forming examples, and regulating your own learning (Davis, 2007).
Active studying does not mean highlighting or underlining text, re-reading, or
rote memorization. Though these activities may help to keep you engaged in the
task, they are not considered active studying techniques and are weakly related
to improved learning (Mackenzie, 1994).

Ideas for active studying include:
%
\begin{itemize}
  \item Create a study guide by topic. Formulate questions and problems and
    write complete answers. Create your own quiz.
  \item Become a teacher. Say the information aloud in your own words as if you
    are the instructor and teaching the concepts to a class.
  \item Derive examples that relate to your own experiences.
  \item Create concept maps or diagrams that explain the material.
  \item Develop symbols that represent concepts.
  \item For non-technical classes (e.g., English, History, Psychology), figure
    out the big ideas so you can explain, contrast, and re-evaluate them.
  \item For technical classes, work the problems and explain the steps and why
    they work.
  \item Study in terms of question, evidence, and conclusion: What is the
    question posed by the instructor/author? What is the evidence that they
    present? What is the conclusion?
\end{itemize}
%
Organization and planning will help you to actively study for your courses. When
studying for a test, organize your materials first and then begin your active
reviewing by topic (Newport, 2007). Often professors provide subtopics on the
syllabi. Use them as a guide to help organize your materials. For example,
gather all of the materials for one topic (e.g., PowerPoint notes, text book
notes, articles, homework, etc.) and put them together in a pile. Label each
pile with the topic and study by topics.
%
\section{Understand the Study Cycle}
The Study Cycle, developed by Frank Christ, breaks down the different parts of
studying: previewing, attending class, reviewing, studying, and checking your
understanding. Although each step may seem obvious at a glance, all too often
students try to take shortcuts and miss opportunities for good learning. For
example, you may skip a reading before class because the professor covers the
same material in class; doing so misses a key opportunity to learn in different
modes (reading and listening) and to benefit from the repetition and distributed
practice that you'll get from both reading ahead and attending class.
Understanding the importance of all stages of this cycle will help make sure you
don't miss opportunities to learn effectively.
%
\section{Spacing out is good}
One of the most impactful learning strategies is “distributed practice”—spacing
out your studying over several short periods of time over several days and weeks
(Newport, 2007). The most effective practice is to work a short time on each
class every day. The total amount of time spent studying will be the same (or
less) than one or two marathon library sessions, but you will learn the
information more deeply and retain much more for the long term—which will help
get you an A on the final. The important thing is how you use your study time,
not how long you study. Long study sessions lead to a lack of concentration and
thus a lack of learning and retention.

In order to spread out studying over short periods of time across several days
and weeks, you need control over your schedule. Keeping a list of tasks to
complete on a daily basis will help you to include regular active studying
sessions for each class. Try to do something for each class each day. Be
specific and realistic regarding how long you plan to spend on each task—you
should not have more tasks on your list than you can reasonably complete during
the day.

For example, you may do a few problems per day in math rather than all of them
the hour before class. In history, you can spend 15-20 minutes each day actively
studying your class notes. Thus, your studying time may still be the same
length, but rather than only preparing for one class, you will be preparing for
all of your classes in short stretches. This will help focus, stay on top of
your work, and retain information.

In addition to learning the material more deeply, spacing out your work helps
stave off procrastination. Rather than having to face the dreaded project for
four hours on Monday, you can face the dreaded project for 30 minutes each day.
The shorter, more consistent time to work on a dreaded project is likely to be
more acceptable and less likely to be delayed to the last minute. Finally, if
you have to memorize material for class (names, dates, formulas), it is best to
make flashcards for this material and review periodically throughout the day
rather than one long, memorization session (Wissman and Rawson, 2012).
%
\section{It's good to be intense}
Not all studying is equal. You will accomplish more if you study intensively.
Intensive study sessions are short and will allow you to get work done with
minimal wasted effort. Shorter, intensive study times are more effective than
drawn out studying.

In fact, one of the most impactful study strategies is distributing studying
over multiple sessions (Newport, 2007). Intensive study sessions can last 30 or
45-minute sessions and include active studying strategies. For example,
self-testing is an active study strategy that improves the intensity of studying
and efficiency of learning. However, planning to spend hours on end self-testing
is likely to cause you to become distracted and lose your attention.

On the other hand, if you plan to quiz yourself on the course material for 45
minutes and then take a break, you are much more likely to maintain your
attention and retain the information. Furthermore, the shorter, more intense
sessions will likely put the pressure on that is needed to prevent
procrastination.
%
\section{Silence isn't golden}
Know where you study best. The silence of a library may not be the best place
for you. It's important to consider what noise environment works best for you.
You might find that you concentrate better with some background noise. Some
people find that listening to classical music while studying helps them
concentrate, while others find this highly distracting. The point is that the
silence of the library may be just as distracting (or more) than the noise of a
gymnasium. Thus, if silence is distracting, but you prefer to study in the
library, try the first or second floors where there is more background ‘buzz.'

Keep in mind that active studying is rarely silent as it often requires saying
the material aloud.
%
\section{Problems are your friend}
Working and re-working problems is important for technical courses (e.g., math,
economics). Be able to explain the steps of the problems and why they work.

In technical courses, it is usually more important to work problems than read
the text (Newport, 2007). In class, write down in detail the practice problems
demonstrated by the professor. Annotate each step and ask questions if you are
confused. At the very least, record the question and the answer (even if you
miss the steps).

When preparing for tests, put together a large list of problems from the course
materials and lectures. Work the problems and explain the steps and why they
work (Carrier, 2003).
%
\section{Reconsider multitasking}
A significant amount of research indicates that multi-tasking does not improve
efficiency and actually negatively affects results (Junco, 2012).

In order to study smarter, not harder, you will need to eliminate distractions
during your study sessions. Social media, web browsing, game playing, texting,
etc. will severely affect the intensity of your study sessions if you allow
them! Research is clear that multi-tasking (e.g., responding to texts, while
studying), increases the amount of time needed to learn material and decreases
the quality of the learning (Junco, 2012).

Eliminating the distractions will allow you to fully engage during your study
sessions. If you don't need your computer for homework, then don't use it. Use
apps to help you set limits on the amount of time you can spend at certain sites
during the day. Turn your phone off. Reward intensive studying with a
social-media break (but make sure you time your break!) See our handout on
managing technology for more tips and strategies.
%
\section{Switch up your setting}
Find several places to study in and around campus and change up your space if
you find that it is no longer a working space for you.

Know when and where you study best. It may be that your focus at 10:00 PM. is
not as sharp as at 10:00 AM. Perhaps you are more productive at a coffee shop
with background noise, or in the study lounge in your residence hall. Perhaps
when you study on your bed, you fall asleep.

Have a variety of places in and around campus that are good study environments
for you. That way wherever you are, you can find your perfect study spot. After
a while, you might find that your spot is too comfortable and no longer is a
good place to study, so it's time to hop to a new spot!
%
\section{Become a teacher}
Try to explain the material in your own words, as if you are the teacher. You
can do this in a study group, with a study partner, or on your own. Saying the
material aloud will point out where you are confused and need more information
and will help you retain the information. As you are explaining the material,
use examples and make connections between concepts (just as a teacher does). It
is okay (even encouraged) to do this with your notes in your hands. At first you
may need to rely on your notes to explain the material, but eventually you'll be
able to teach it without your notes.

Creating a quiz for yourself will help you to think like your professor. What
does your professor want you to know? Quizzing yourself is a highly effective
study technique. Make a study guide and carry it with you so you can review the
questions and answers periodically throughout the day and across several days.
Identify the questions that you don't know and quiz yourself on only those
questions. Say your answers aloud. This will help you to retain the information
and make corrections where they are needed. For technical courses, do the sample
problems and explain how you got from the question to the answer. Re-do the
problems that give you trouble. Learning the material in this way actively
engages your brain and will significantly improve your memory (Craik, 1975).
%
\section{Take control of your calendar}
Controlling your schedule and your distractions will help you to accomplish your
goals.

If you are in control of your calendar, you will be able to complete your
assignments and stay on top of your coursework. The following are steps to
getting control of your calendar:
%
\begin{enumerate}
  \item On the same day each week, (perhaps Sunday nights or Saturday mornings)
    plan out your schedule for the week.
  \item Go through each class and write down what you'd like to get completed
    for each class that week.
  \item Look at your calendar and determine how many hours you have to complete
    your work.
  \item Determine whether your list can be completed in the amount of time that
    you have available. (You may want to put the amount of time expected to
    complete each assignment.) Make adjustments as needed. For example, if you
    find that it will take more hours to complete your work than you have
    available, you will likely need to triage your readings. Completing all of
    the readings is a luxury. You will need to make decisions about your
    readings based on what is covered in class. You should read and take notes
    on all of the assignments from the favored class source (the one that is
    used a lot in the class). This may be the textbook or a reading that
    directly addresses the topic for the day. You can likely skim supplemental
    readings.
  \item Pencil into your calendar when you plan to get assignments completed.
  \item Before going to bed each night, make your plan for the next day. Waking
    up with a plan will make you more productive.
\end{enumerate}
%
\section{Use downtime to your advantage}
Beware of ``easy'' weeks. This is the calm before the storm. Lighter work weeks
are a great time to get ahead on work or to start long projects. Use the extra
hours to get ahead on assignments or start big projects or papers. You should
plan to work on every class every week even if you don't have anything due. In
fact, it is preferable to do some work for each of your classes every day.
Spending 30 minutes per class each day will add up to three hours per week, but
spreading this time out over six days is more effective than cramming it all in
during one long three-hour session. If you have completed all of the work for a
particular class, then use the 30 minutes to get ahead or start a longer
project.
%
\end{document}
