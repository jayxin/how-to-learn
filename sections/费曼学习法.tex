\documentclass[../main.tex]{subfiles}
\graphicspath{{\subfix{../figures/}}}
%
\title{费曼学习法}
%
\begin{document}
\maketitle
学习是一件痛苦的事情,学而不得尤为痛苦。
谈到学习,很多人可能都质问过自己:
%
\begin{itemize}
  \item 为什么别人读书可以信手拈来,我却不行?
  \item 为什么我学了很多知识却没有办法去实际应用它?
  \item 面对复杂的知识和概念我为什么不能理解和吸收?
\end{itemize}
%
面对以上难题,今天和大家分享下这个被称为``终极学习方法''的高效技巧:
\textbf{费曼技巧}。
%
\section{原理}
\begin{cuenotes}
  \cue{什么是费曼技巧?}
  \note{
    费曼技巧是一种``以教为学''的学习方式,
    能够帮助你提高知识的吸收效率,真正理解并学会运用知识。
    这个学习方法其实很简单,就是验证你是否真正掌握一个知识,
    看你能否用直白浅显的语言把复杂深奥的问题和知识讲清楚。
  }
  \cue{具体应用方式}
  \note{
    \begin{enumerate}
      \item 向不熟悉某议题的人解释该议题,
        用他们能理解的方式及最简单的语言向他们解释;
      \item 发现自己不能理解的地方或不能简单解释某议题的地方并记录;
      \item 回头查看资讯来源并研读自己薄弱的地方直到能用简单的语言来解释;
      \item 重复前面三项步骤直到能够专精这个议题.
    \end{enumerate}
  }
  \cue{使用费曼技巧的效果}
  \note{
    \begin{itemize}
      \item 真正地了解任何你学习的事物
      \item 做出深思熟虑并有智慧的决定
      \item 熟练地将知识应用到实际问题
    \end{itemize}
  }
\end{cuenotes}
%
\section{为什么费曼技巧如此高效}
\subsection{拆分和压缩知识}
\textbf{拆分问题}:当你想了解一个复杂的知识点时,需要把它分而化之,
切成小知识块,再逐个对付。

\textbf{压缩知识}:一本书很厚,里面的信息容量很大,
我们无法记住所有的内容。但聪明的人会把书本呈现的信息进行压缩,
提炼出规律和知识,来达到和原有的知识体系产生联系。压缩知识的过程,
也是理解和内化的过程。
%
\subsection{理解和简化知识}
为什么很多人不会运用知识,无法做到举一反三?
很大原因是因为你没有真正地去理解知识。要理解一个复杂问题,
你需要调用自身的知识储备。比如要理解``沉没成本''这个概念,
你需要了解一点经济学和商业知识背景。
这就要求你``回头查看资讯来源并研读自己弱点的地方''。

当你真正理解这个概念后,要如何把这个知识传授给没有相关知识背景的人?
举例子,简化知识,把复杂的知识用简单的例子来进行说明。

举例子是一个能增进专精程度并学习同理的好方法,
这促使你用对方的程度来理解并透过与他们熟悉的议题有关的方式给予他们新的知识。
你在理解和简化知识的过程中会不知不觉用到类比、举例、概括、对比等
深度学习的方法。
%
\subsection{理解 $ \iff $ 复述}
如果你不能简单地解释一件事,那你就还没有弄懂它。
很多时候我们自以为已经掌握了某个知识,但其实并没有。
如果你不能讲清楚,也就说明你没有掌握。这时候你就需要更深度地了解知识。
理解和复述是相互促进的作用,费曼技巧就是在不断强化这个过程。
%
\end{document}
