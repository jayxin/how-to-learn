\documentclass[../main.tex]{subfiles}
\graphicspath{{\subfix{../figures/}}}
%
\title{Learning A Second Language}
%
\begin{document}
\maketitle
%
\section{Introduction}
Learning a language is a complex, time-intensive task that requires dedication,
persistence, and hard work. If you're reading this, then you probably already
know that.

What you might not know is that there are strategies that can help you study
more effectively, so that you make the most of your time and energy. This
handout first explains some of the key principles that guide effective language
learning, and then describes activities that can help you put these principles
into practice. Use these tools to create a strategic study plan that helps your
language skills grow.
%
\section{Key principles of language learning}

\textbf{The Basics}: \\
First, let's talk about the basics. Research in this area (called ``second
language acquisition'' in academia) suggests that there are three key elements to
learning a new language.
%
\begin{itemize}
  \item The first is comprehensible input, which is a fancy way of saying being
    exposed to (hearing or reading) something in the new language and learning
    to understand it.
  \item Comprehensible output is the second element, and unsurprisingly it means
    learning to produce (speak or write) something in the new language.
  \item The third element is review or feedback, which basically means
    identifying errors and making changes in response.
\end{itemize}
%
Fancy terms aside, these are actually pretty straightforward ideas.

These three elements are the building blocks of your language practice, and an
effective study plan will maximize all three. The more you listen and read
(input), the more you speak and write (output), the more you go back over what
you've done and learn from your errors (review \& feedback), the more your
language skills will grow.

\textbf{DO}: Create a study plan that maximizes the three dimensions of language
learning: understanding (input), producing (output), and identifying and
correcting errors (review/feedback).
%
\subsection{Seek balance}
Learning a new language involves listening, speaking, reading, writing,
sometimes even a new alphabet and writing format. If you focus exclusively on
just one activity, the others fall behind.

This is actually a common pitfall for language learners. For example, it's easy
to focus on reading comprehension when studying, in part because written
language is often readily accessible—for one thing, you have a whole textbook
full of it. This is also true of the three key elements: it's comparatively easy
to find input sources (like your textbook) and practice understanding them. But
neglecting the other two key principles (output and feedback/review) can slow
down language growth.

Instead, what you need is a balanced study plan: a mix of study activities that
target both spoken and written language, and gives attention to all three key
principles.

\textbf{DO}: Focus on balance: practice both spoken and written language, and
make sure to include all of the three key principles—input, output, and
feedback/review.
%
\subsection{Errors are important}
Sometimes, the biggest challenge to language learning is overcoming our own
fears: fear of making a mistake, of saying the wrong thing, of embarrassing
yourself, of not being able to find the right word, and so on. This is all
perfectly rational: anyone learning a language is going to make mistakes, and
sometimes those mistakes will be very public.

The thing is, you NEED to make those mistakes. One of the key principles of
language learning is all about making errors and then learning from them: this
is what review \& feedback means. Plus, if you're not willing to make errors,
then the amount of language you produce (your output) goes way down. In other
words, being afraid of making a mistake negatively affects two of the three key
principles of language learning!

So what do you do? In part, you may need to push yourself to get comfortable
with making errors. However, you should also look for ways to get low-stakes
practice: create situations in which you feel more comfortable trying out your
new language and making those inevitable mistakes.

For example, consider finding a study partner who is at your level of language
skill. This is often more comfortable than practicing with an advanced student
or a native speaker, and they're usually easier to find—you've got a whole class
full of potential partners!

\textbf{DO}: Learn to appreciate mistakes, and push yourself to become more comfortable
with making errors.

\textbf{DO}: Create opportunities for ‘low-stakes' practice, where you'll feel
comfortable practicing and making mistakes.
%
\subsection{Spread it out}
Studying a new language involves learning a LOT of material, so you'll want to
use your study time as effectively as possible. According to research in
educational and cognitive psychology, one of the most effective learning
strategies is distributed practice. This concept has two main components:
spacing, which is breaking study time up into multiple small sessions, and
separation, which means spreading those sessions out over time.

For example, let's imagine you have a list of vocabulary words to learn. Today
is Sunday, and the vocab quiz is on Friday. If you can only spend a total of 30
minutes studying this vocab, which study plan will be the most effective?
%
\begin{itemize}
  \item Study for 30 minutes on Thursday.
  \item Study for 10 minutes at a time on Tuesday, Wednesday, and Thursday.
  \item Study for 10 minutes at a time on Sunday, Tuesday, and Thursday.
  \item Study for 30 minutes on Sunday.
\end{itemize}
%
If you look at the total time spent studying, all four options are exactly the
same. But research suggests that option C is the most effective way to manage
your time: instead of studying the vocabulary all at once, you've spread out the
time into several shorter sessions, and you've also increased the amount of time
between study sessions. (And yes, this is also why ``cramming'' isn't a good study
plan!)

\textbf{DO}: Break up your study time into shorter chunks and spread those
sessions out over time.
%
\subsection{Bump up your memory}
Memory is a critical part of any kind of studying, and effective memorization is
strongly correlated with success in foreign language classes. But if you're
not ``good at'' memorizing things, don't despair! Although people often think of
memory as a fixed quality, it's actually a skill that you can improve through
deliberate practice.

There's a considerable amount of research on how memory works, as well as a wide
range of strategies for improving memory. For example, scientific experiments
show that our short-term memory can only hold about 7 pieces of new information
at once. So if you're working on a long list of new vocabulary words, start by
breaking it up into smaller chunks, and study one shorter section at a time.
Additionally, research also suggests that recall-based study methods are most
effective. This means that actively trying to recall information is more
effective than simply reviewing information; essentially, self-testing will help
you more than re-reading your notes will.

The best way to start working on your memory is to build on the techniques that
you already know work for you. For example, if associating a word with a picture
is effective for you, then you should incorporate images into your vocabulary
practice. However, if you're not sure where to start, here's a ``beginner''
formula for memorizing a new word: use the word at least five times the first
day that you learn it, then multiple times over the week, at least once every
day.

In addition to figuring out which memorization techniques work best for you,
it's also important to actively protect your memory. For example, experiencing a
strong emotion has been shown to sharply decrease the ability to memorize
unrelated content. (So if you've just watched a horror movie, it's probably not
a great time for vocabulary review!)

To get the most out of your study time, here's a list of common ``memory killers''
to avoid:
\begin{itemize}
  \item \textbf{Stress and anxiety}: Just like other strong emotions, stress and
    anxiety drastically reduce your ability to make new memories and recall
    information.
  \item \textbf{Information overload}: Studying for hours at a time might seem
    like a great idea, but it's actually a really ineffective use of time. In
    fact, taking a short break every 30 minutes helps improve focus, and after 2
    hours you should consider switching topics.
  \item \textbf{Fatigue}: The more tired you are, the less effective your memory
    is. Chronic sleep deprivation is particularly detrimental, so those
    late-night study sessions might actually do more harm than good!
  \item \textbf{Multitasking}: As you may have noticed, all of these ``memory
    killers'' are also things that disrupt focus. Multi-tasking is probably the
    most common source of distraction. In fact, here's a great rule of thumb for
    protecting your memory: if you're not supposed to do it while driving, then
    you shouldn't do it while studying. (Yes, that means drinking, texting, and
    watching Netflix ``in the background'' are all NOs.)
\end{itemize}

\textbf{DO}: Increase memorization by breaking information into small chunks and
studying the chunks one at a time, and by using recall-based strategies like
self-testing.

\textbf{DO}: Focus on protecting and improving your memorization skills, and
build the memory techniques that work best for you into your study plan.
%
\subsection{Vocab is king}
Want to know a secret? Vocabulary is more important than grammar.
DISCLAIMER: This does NOT mean that grammar is unimportant. Without grammar, you
won’t know how to use your vocabulary, since grammar tells you how to combine
words into sentences. And obviously, if you’re in a foreign language class,
you’re going to need to study ALL the material to do well, and that will
definitely include grammar.

The more vocabulary you know, the more quickly you can grow your language
skills. The reason is simple: understanding more words directly translates into
more input, producing more words means more output, and more output means more
opportunity for feedback. Additionally, when you’re interacting with native
speakers, vocabulary is more beneficial to communication than grammar is. Being
able to produce words will help get your meaning across, even if what you say is
not perfectly grammatical.

Of course, in order to become fully fluent in your new language, eventually you
will need strong grammar skills. But once again, this is something that having a
strong, well-developed vocabulary will help with. Since grammar dictates
relationships between words and phrases, understanding those smaller components
(aka vocabulary) will help improve your understanding of how those grammatical
relationships work.

\textbf{DO}: Design a study plan that emphasizes vocabulary.
%
\section{Activities}
Now that we’ve talked about the general principles that you should incorporate
in your language study, let’s focus on activities: practical suggestions to help
you find new ways to grow your language skills!
%
\subsection{Find real-life sources}
Since one of the main 3 components of language learning is input, look for ways
to expose yourself to as much of the language you’re learning as possible. But
this doesn’t mean reading more textbooks (unless your textbook is a fascinating
read that you’re excited about). Instead, look for “authentic” examples of the
language, things you’ll actually enjoy and look forward to practicing with, even
if you don’t understand every word!

Here are some examples to get you started:
%
\begin{itemize}
  \item Newspaper articles, magazines, \& blogs: Many of these are freely
    available online, and once you’ve tried reading them a few times, it’s easy
    to translate the key parts to check your understanding. Look for a topic
    you’re already interested in and follow it with a news reader app!
  \item Books: Children’s picture books and books you’ve read before in your
    native language are easy options for intermediate/advanced beginners. The
    library often has great options available for free!
  \item TV shows and movies: Try watching them without subtitles the first time,
    starting in ~15 minute segments. Another great option is to watch first
    without any subtitles, then with subtitles in the language you’re learning,
    and then finally with subtitles in your native language if you need them.
    Soap operas are also great options (especially if you like lots of drama!),
    since the plot lines are often explained multiple times.
  \item Songs: Music, especially popular songs, can be especially well suited to
    language practice, since you’re likely to memorize the ones you enjoy. Ask a
    teacher or native speaker for recommendations if you’re struggling to find
    good examples. Children’s songs can also be fun practice tools.
  \item Podcasts and audio books: There are a lot of options for all sorts of
    languages, and as a bonus you’ll often get exposure to local news and
    cultural topics.
\end{itemize}
%
Also, consider tweaking some of your media settings to “bump up” your casual
language exposure. For example, changing your Facebook and LinkedIn location and
language preferences will force you to interact with the language you’re
learning, even when you’re (mostly) wasting time.
%
\subsection{Pro tips}
Improve the effectiveness of this activity by using the following suggestions!
%
\begin{itemize}
  \item Slow it down: If you’re listening to a podcast or audio book, try
    slowing down the speed just a bit: 0.75x is a common option, and the
    slowed-down audio still doesn’t sound too strange. Also, make sure to take
    breaks frequently to help you process what you’ve just heard.
  \item Combine your senses: In many cases, you can combine types of input to
    help create a more learning environment: reading and listening to a text at
    the same time can help you improve your comprehension. For example, for TV
    shows and movies, turn on subtitles in the same language. Other options
    include:
    \begin{enumerate}
      \item Radio news stories often have both audio and transcripts available
        online, especially for pieces that are a few days old.
      \item Amazon’s Kindle offers an “immersive reading” option that syncs
        audio books with text.
      \item TED talks come in many different languages, and often include an
        interactive transcript.
    \end{enumerate}
  \item Get hooked: To make this strategy as effective as possible, find a
    source that you really enjoy, and commit to experiencing it only in the
    language you’re learning. Having a go-to program that you love will help
    keep you motivated. For example, if you love podcast/radio story programs
    like “Radiolab” and are learning Spanish, check out “Radio Ambulante.”
\end{itemize}
%
\subsection{Hold shadow conversations}
A key part of learning a new language involves training your ear. Unlike written
language, spoken language doesn’t have the same context clues that help you
decipher and separate out words. Plus, in addition to using slang and idioms,
native speakers tend to “smoosh” words together, which is even more confusing
for language learners! In part, this is why listening to real-life sources
can be so helpful (see the previous activity).

However, even beginning language learners can benefit from something called
conversational shadowing. Basically, this means repeating a conversation
word-for-word, even when you don’t know what all of the words mean. This helps
you get used to the rhythm and patterns of the language, as well as learn to
identify individual words and phrases from longer chunks of spoken language.
Another great strategy involves holding practice conversations, where you create
imaginary conversations and rehearse them multiple times.

Both of these strategies are great ways to help you learn and retain new
vocabulary, and they also increase your language output in a low-stakes practice
setting!

\textbf{Example}: If you’ve got a homework exercise that involves reviewing an
audio or video clip, take a few extra steps to get the most benefit:
%
\begin{itemize}
  \item After you’ve listened to the clip once, shadow the conversation in short
    sections (think ~20-30 seconds). Focus on reproducing the words as
    accurately as possible, paying close attention to rhythm, intonation, and
    pacing.
  \item Once you can accurately shadow the entire clip, then focus on
    understanding the meaning of the material, and answer any homework questions
    related to the clip.
  \item Now, use the same vocabulary to create a new conversation: think of what
    you would want to say in a real-life situation like this one, and practice
    it until you can respond confidently to any side of the exchange.
\end{itemize}
%
\subsection{Become a collector}
Since expanding your vocabulary is so important, identifying new words is a big
priority. This is especially true when you’re in an immersion environment
(studying abroad, etc), but it’s also something that you can do on a regular
basis even when you’re at home.

Basically, you need to collect words: any time you encounter a new word, you
want to capture it by recording it in some way. The easiest way to do this is in
a small pocket notebook, but you could also put a note in your phone, send a
text or email to yourself, or even record yourself saying it. The key point is
to capture the word as quickly and easily as possible. Also, don’t worry too
much about spelling or definitions in the moment: you’ll deal with those later.

Whatever your recording system is (notebook, phone, voice memo, etc), it’s only
the first part of the collection process. Next, you’ll need to review each of
the words you’ve recorded. This is something you’ll do on a regular basis, so
that you can actually use the words you’ve recorded. Depending on how many new
words you’re collecting, it might be every day, every few days, or once a week.
This is the time when you find the correct spelling, write down the definition,
maybe find an example, and so on.

To make this process as effective as possible, you also want to have some sort
of system that helps you record and organize your word collection. If you like
paper-based methods, then flashcards can be easily organized in index card
boxes, though you might want to include some alphabetical divider tabs to help
yourself stay organized. However, digital tools are particularly helpful with
this kind of information, and there are tons of apps that can help you organize
a large vocabulary collection. But you don’t need a fancy app or program: a
simple spreadsheet also works great for most cases.

Finally, you also want to make sure to use your word collection! Not only do you
need to learn new words once you add them, you’ll also need regular review of
old words to maintain your vocabulary. This is another place where digital tools
shine, since it’s easy to access the entire collection at any time, making it
easier to study and review on a regular basis. In any case, make sure that you
incorporate review along with learning new words.
%
\section{The 4 basic steps of word collection}
\begin{enumerate}
  \item Capture new words. Listen for them in class, seek them out in
    conversations, find them in your “authentic sources,” etc. Record them in
    the moment, without worrying too much about spelling and definitions.
  \item Review your new words. Establish a routine so that you regularly “empty
    out” your recording tool and add the new words to your collection.
  \item Record and organize your collection. Create an organized system for your
    collection; common tools include digital flashcard apps, spreadsheets, and
    traditional index cards.
  \item Use your words! Make sure you’re learning new additions and also
    periodically reviewing older words.
\end{enumerate}
%
\subsection{Pro tips}
If you’re struggling to find new words to collect—or if you feel overwhelmed by
the number of words you could collect—then try working “backwards.” Instead of
looking for new words in the language you’re learning, think about the gaps in
your vocabulary. For example, think about the topics you frequently discuss in
your native language. Do you know how to talk about those things in the language
you’re learning? Hobbies and other classes are often great places to start.

If you’re in a foreign language class, you can use the same word collection
system to help you learn and review assigned vocabulary. Consider color-coding
or tagging words that are class-related if you want to give those words extra
attention. If you’re using a digital flashcard app, you might consider creating
different card “sets” to help you organize them.
%
\section{Flashcard zen}
Flashcards are one of the most common tools that language learners use. There is
a good reason for this: they’re easily portable, they’re excellent for learning
short pieces of information (like new words), and used correctly they’re a great
recall-based study strategy. However, flashcards are not without problems. For
example, it’s far too easy to devote excessive time to making elaborately
detailed flashcards, and then spend comparatively little time actually using
them! The following tips describe ways to use flashcards in a strategic and
effective manner.
%
\subsection{Less is more}
The more time you spend making flashcards, the less time you spend using
them…but if you don’t make flashcards, then you don’t have any to use. The point
behind this paradox is that you want to minimize the time and effort you put
into the flashcard set-up process. This is a situation where perfectionism can
really harm you: if you focus on making absolutely “perfect” flashcards, then
you’re really just wasting time.

Similarly, you also want to minimize the volume of information you put on each
flashcard. Flashcards should not be pages of notes in a smaller format,
especially when using them for vocabulary. Instead, each card should have just
enough information on it to test your memory. Instead of containing many
details, a good flashcard will serve as a “cue” that triggers your memory. This
way, you’re forcing your brain to work to produce the information, which helps
build and maintain strong memories.
%
\subsection{Mix it up}
Another common flashcard issue is that they promote rote memorization, so that
information is divorced from context. But in real life, you’ll be using your
vocabulary in a wide range of contexts. Only practicing vocabulary in rote
drills may end up slowing you down when you need to actually use the words.

One example of this is the “translation” phenomenon: instead of learning to
associate new words with their meanings, they become associated with the word in
your native language. If you’re always translating word-for-word in your head,
then it takes much longer to understand and interact. A great way to reduce this
issue is to change the type of cues used on your flashcard: instead of written
words, you might represent the meaning of new words with a picture—or for
digital flashcards, you could even use audio files.

\textbf{Example}: Imagine a beginning student (and native English speaker)
learns that the Arabic word for door is “bab” . She could make several
different flashcards for this word:
%
\begin{itemize}
  \item Traditional flashcard: the written word in Arabic on one side, and in
    English on the other
  \item Audio flashcard (digital): the spoken Arabic word on one side, and the
    spoken word in English on the other
  \item Pictoral flashcard: a picture of a door on one side, and the word
    written in Arabic on the other
\end{itemize}
%
You can also combine these types to make different hybrid-style flashcards. Once
again, don’t try to make elaborate, perfect flashcards—just something that will
push you to associate words with meanings, instead of just their translations.
Not all of your flashcards have to use non-written cues, but it’s a great way to
add variety and prevent “translation” memory.

Additionally, make sure to practice using both sides of the flashcards as cues.
In other words, if you’ve already gone through a set of cards starting with the
English side, flip the stack over the next time you use it, so that you’re
getting prompted by the language you’re learning.

You can also avoid the pitfalls of rote memorization by making sure to practice
using the words in context. For example, in addition to testing yourself with
each card, follow that up by using the word in a sentence. This is particularly
good for words you’ve already learned and are now reviewing. You can also turn
this into a game, where you make up “mad-lib” style sentences by randomly
drawing cards and combining them. If you’re working with a partner or study
group, you can also use flashcards to play games like charades or Pictionary.
%
\subsection{Make it a habit}
Ultimately, flashcards are just a tool, albeit one that is ideally suited to
vocabulary practice. And as with any kind of practice, the more time you put in,
better your results will be: flashcards work best when used frequently and
consistently. If you want to get the most out of your flashcards, turn using
them into a regular habit. Here are some tips to keep in mind:
\\

\begin{cuenotes}
  \cue{Small sets, many reps}
  \note{
    To improve memorization when practicing new words, create sets of 7
    flashcards or less and practice each set several times before moving on to
    the next one. Also, make sure to space out your flashcard sessions, and once
    you’ve reviewed a set of words, put it aside for a day or two before
    reviewing it again.
  }
  \cue{Increase portability}
  \note{
    Make sure you take full advantage of the portable nature of flashcards. If
    you’re using paper, then consider using a binder ring and hole punch to keep
    small sets together. You might also use smaller cards: since you’ll be
    making simple cards (minimalism!), you could probably cut a regular 3”x5”
    index card into halves (or even quarters) and still have more than enough
    room! Even if you use full-sized paper cards, you increase portability by
    being selective in the number you take with you. Remember, you want to space
    out your sets and reps, so it isn’t necessary to carry all your cards with
    you all the time. If you’re using digital tools, look for apps that can sync
    to all your devices—phone, tablet, computer, web, etc.
  }
  \cue{Wasted’ time}
  \note{
    Since flashcards are so portable, they’re a great way to turn “wasted” time
    into useful time. How much time do you spend riding the bus? How about stuck
    in line at the grocery store, or waiting for an appointment at Campus
    Health? Instead of checking your Twitter feed or hopping on Facebook, open
    up your flashcard app (or pull out your flashcard stack) and do a quick
    vocab review. If you’re doing small sets it won’t take very long to go
    through one, and you’ve just bumped up your number of reps for the day!
  }
  \cue{Create a routine}
  \note{
    Habits are powerful. Once you’ve established a behavior pattern, you find
    yourself doing it without thinking about it. So think about how you can
    create a daily routine for using your flashcards. Finding and using “wasted”
    time is a good start, especially if you have a daily bus commute. What about
    taking 5 minutes every morning to do vocabulary review while you drink your
    coffee? Or making it your first “after-dinner to do” once you’ve finished
    eating? Once you find ways to make vocabulary flashcards part of your daily
    routine, you can use the power of habit to help grow your vocabulary.
  }
  \cue{Periodic review}
  \note{
    Once you’ve learned new words, you’re not done with those
    flashcards—instead, use them to keep your vocabulary strong. Each week,
    randomly select a few words to review. You might do a review set once each
    day, or the review words can be mixed in with your current learning sets
    (this is a great way to keep your word collection going strong!).
  }
\end{cuenotes}
%
\section{Make it fun}
Learning a new language is a lot of work, but that’s not what motivated you to
start studying it in the first place, right? Instead, you probably want to
travel or work abroad, or be able to talk with people from other countries,
maybe even study literature or history $ \ldots $ Whatever got you interested in this
language in the first place, it’s probably a lot more fun than all this studying
is.

Here’s the thing: whenever you can do something that connects you back with the
reasons that motivate you to study your new language, or you find something new
and exciting about the language you’re studying or the cultures that use it, use
your excitement to boost your motivation. It’s what will keep you going—and that
kind of persistence is a key factor in language learning success.

But in addition to staying focused on what you enjoy, you can also deliberately
create fun social activities that also help you grow your language skills. For
example, try hosting a dinner and movie “theme” night with friends who are
studying the same language. Create a “mini-immersion” environment: watch movies
in the language you’re learning, cook some authentic cuisine, and try to speak
only in your new (shared!) language. It’s a great way to get some authentic,
low-stakes practice. (Plus, it’s a great excuse for a party!)
%
\end{document}
