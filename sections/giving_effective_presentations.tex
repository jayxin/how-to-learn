\documentclass[../main.tex]{subfiles}
%
\title{Giving Effective Presentations}
%
\begin{document}
\maketitle
%
\section{Before the presentation}
Determine the type of speech delivery you are giving. The type of speech
delivery influences the strategies you will use to practice the speech. The four
types of speech delivery are:
\\

\begin{cuenotes}
  \cue{Impromptu}
  \note{
    A speech that has no advanced planning or practice.
  }
  \cue{Extemporaneous}
  \note{
    The speaker prepares notes or an outline, with embellishment. This kind of
    speech allows the speaker to adapt to the audience’s reaction and sounds
    more natural and conversational than scripted speeches.
  }
  \cue{Manuscript}
  \note{
    Reading a scripted speech word-for-word without any memorization.
  }
  \cue{Memorized}
  \note{
    Memorizing a scripted speech to present without having to rely on reading
    the script.
  }
\end{cuenotes}

When thinking about how you will deliver your speech, consider:
\begin{cuenotes}
  \cue{Articulation}
  \note{
    Find the right pace for your speech in order to retain clarity and be easily
    understood by your audience.
  }
  \cue{Nonverbal communication}
  \note{
    This can include posture, eye contact, facial expressions, and movement that
    can be used to reinforce or modify your speech.
  }
  \cue{Effective voice}
  \note{
    Strive for a conversational, casual voice at a volume that your audience can
    comfortably hear even if they are listening from the back of the room or
    through a digital platform like Zoom.
  }
\end{cuenotes}

Below are some simple steps to take in practicing for your speech or presentation:
\\

\begin{cuenotes}
  \cue{Practice your stance}
  \note{
    If you will be standing while presenting your speech, then practice while
    standing. If you’ll be seated, practice while seated. If you’ll be on Zoom,
    practice to see what stance will work for your camera.
  }
  \cue{Speak out loud}
  \note{
    It can be tempting to recite quietly when preparing. But reciting your
    speech at a similar volume to the volume you will use to present in class is
    another effective method of practice.
  }
  \cue{Practice without looking at the text}
  \note{
    Even if you are not required to memorize your speech or presentation,
    practicing without looking at your text to see if you lose your train of
    thought.
  }
  \cue{Make your script easy to read}
  \note{
    If you are planning on reading your speech from a script or notecards,
    format the words to make them easy to reference while you are presenting.
    Try these formatting tips:
    \begin{itemize}
      \item Use a large font (14 pt. or greater) for text and numbering your
        pages.
      \item Leave blank lines between paragraphs or sentences so you can easily
        find your place.
      \item Refrain from stapling pages together or printing double sided, which
        makes your pages more difficult to organize during the presentation.
      \item Highlight words or lines that are important to you to ensure you
        deliver them during the presentation.
      \item For in class speeches, copy down presenter notes on paper from any
        PowerPoint or presentation slides in order to refrain from staring at a
        screen during your presentation.
    \end{itemize}
  }
  \cue{Record yourself and listen to the recording}
  \note{
    Recording yourself provides a way for you to gain an outside perspective of
    your presentation. Create a list of items you want to improve upon for your
    next trial run and practice again.
  }
  \cue{Ask yourself questions}
  \note{
    The next step in building confidence in presenting is to evaluate your
    progress in being precise. Ask yourself:
    \begin{itemize}
      \item Where did I have trouble speaking clearly and/or emphatically?
      \item Did I stay within my time limit?
      \item Do I feel the need to delete or edit anything?
      \item At what point did I feel the most confident?
    \end{itemize}
  }
  \cue{Practice in front of an audience}
  \note{
    Grab a friend, group member, family member or make an appointment with an
    Academic Coach and recite your speech or presentation in front of another
    person. This will help you bounce ideas off of someone and give you the
    chance to practice in front of people with whom you feel comfortable. For
    Zoom presentations, go through the mechanics of an actual Zoom rehearsal
    with another person. This may involve additional steps of preparing the
    physical space, such as lighting, tidying up, and setting the scene that
    will be viewable by the audience.
  }
  \cue{Try the Wonder Woman pose}
  \note{
    It may feel funny, but some studies suggest that standing with your legs
    apart, hands on your hips, and chest out (much like Wonder Woman) for two
    minutes before a stressful event can build feelings of confidence.
  }
\end{cuenotes}
%
\section{During the presentation}
When it comes to the day of the presentation, it is natural to experience
feelings of nervousness or anxiety, but remember you have practiced for this
presentation and be confident in what you have accomplished. Here are some tips
on how to calm nerves in the moment and manage anxiety:
\\

\begin{cuenotes}
  \cue{Arrive prepared}
  \note{
    Make sure you have everything you need to give the presentation. Using your
    computer? Make sure it’s charged. Are you reading off a script or notecards?
    Pack them with your things the night before.
  }
  \cue{Breathing exercises}
  \note{
    Try 4-7-8 breathing to slow down and calm the mind and body. Inhale quietly
    through your nose to the count of four; hold your breath to the count of
    seven; exhale with sound through your mouth to the count of eight.
  }
  \cue{Maintain eye-contact}
  \note{
    This will help you stay engaged with your audience and hold their attention
    during the speech. The goal is to make eye contact or maintain the illusion
    of eye-contact with members in the audience at all times you are speaking to
    them. An effective strategy for maintaining eye contact is to shift your
    focus between different areas of the audience (room) every few seconds. On
    Zoom, you can look at your camera to simulate the appearance of eye contact
    for the audience rather than looking at the faces on the screen.
  }
  \cue{Incorporate movement}
  \note{
    Consider incorporating movement into your presentation. Movement can help
    support your message, connect with the audience, and dissipate nervous
    energy. Some tips for incorporating body movement include:
    \begin{itemize}
      \item Moving through transitions. For instance, stand firm when delivering
        a point, move to a separate place as you transition to another point,
        and stand firm when delivering your second point).
      \item Stepping forward when delivering a key takeaway message.
      \item Walking toward the audience during a participatory part of your
        presentation, for instance, a Q\&A session.
    \end{itemize}
  }
  \cue{Engage your audience}
  \note{
    Present in a way that is engaging through:
    \begin{itemize}
      \item Knowing your audience
      \item Pacing yourself
      \item Using a sense of humor
      \item Starting off strong
      \item Asking questions or use anecdotes to peak interest
    \end{itemize}
  }
\end{cuenotes}
%
\end{document}
