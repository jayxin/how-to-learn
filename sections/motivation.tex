\documentclass[../main.tex]{subfiles}
%
\title{Motivation}
%
\begin{document}
\maketitle
%
\section{Introduction}
Do you ever procrastinate to avoid unpleasant tasks or assignments? Do you find
it hard to get started? Do you struggle to stay focused and on task when working
from your dorm room, house, or apartment? Motivating yourself to go to class,
complete assignments, study, and do all the other things required of you in
college can be difficult---but it’s crucial to your college success. Research
shows that students can learn how to become better learners by using effective
motivation strategies. Successful students know how to self-regulate (control)
their own learning and the factors that impact their learning. Fortunately,
there are strategies for increasing motivation and self-efficacy, which can in
turn increase chances of academic success and well-being.

This handout explores common challenges when it comes to accomplishing tasks and
shares several tips and strategies to improve your self-motivation.
%
\section{Effort over ability}
One of the key differences between people who do and don’t succeed is not their
ability level but their effort and motivation levels. Few people wake up wanting
to do unpleasant or boring tasks. The ones who do them and succeed in them are
the ones who believe they can and motivate themselves to do them even when they
don’t feel like it. Here are some specific strategies you can use to develop
your self-motivation and improve your overall success.
%
\section{Motivational strategies}
\subsection{Strategies to set yourself up for success}
\begin{cuenotes}
  \cue{Set clear goals.}
  \note{
    Include daily, weekly, semester, and long-term goals. Write them down
    somewhere easily visible. Use SMART goals to be specific and create a plan:
    Specific, Measurable, Achievable, Realistic, Time-bound. For example:
    Instead of saying, “I want to get better grades,” say something like, “I
    want to get at least an 85\% on my BIO exam on March 5.” Even better, set up
    concrete goals (e.g., increased study hours, peer tutoring) that help you
    track your progress toward that long-term goal. For example, instead of
    saying, “I want to do well in my online classes,” say something like, “I
    want to devote thirty minutes tomorrow to taking notes on my Spanish
    textbook before starting my online homework.”
  }
  \cue{Help yourself focus.}
  \note{
    Eliminate or limit things that are distracting and cause you to
    procrastinate. Take distracting apps off your phone, turn off the TV, study
    outside of your dorm room, keep your phone/laptop away during class or study
    times, create a designated study space in your bedroom or home, block
    Netflix, clear out the junk food, etc. If you know you struggle with
    something, make it more difficult for you to indulge in that temptation.
  }
  \cue{Pace yourself.}
  \note{
    Chunk your study, work, and reading times into small sections (30-60
    minutes) with breaks in between. Breaks are important for your focus,
    health, and motivation and should be worked into any study time. If you are
    going to be studying or working for longer, go back and forth between one
    task or class and another.
  }
  \cue{Prioritize.}
  \note{
    Study early in the day and do the most challenging or unpleasant tasks
    first. Research shows that tackling difficult tasks first thing in the day
    can make you feel better throughout the rest of the day and be more
    productive. Doing so keeps you from procrastinating all day and having that
    dreaded feeling of knowing that you need to do something unpleasant.
  }
  \cue{Location, location, location.}
  \note{
    Think about where you work best and where you will be most motivated to get
    to work and stay working. For most people, their dorm room or bed are not
    ideal, as they come with many distractions. Some students focus better in a
    public place like the library or a coffee shop, while others prefer silence
    and isolation, like a quiet and secluded room on campus. Some students
    benefit from blocking off an area in their home that they use exclusively
    for studying and working on projects.
  }
\end{cuenotes}
%
\subsection{Self-care strategies}
\begin{cuenotes}
  \cue{Get enough sleep.}
  \note{
    Aim for at least 7 hours a night. Sleep is important to motivation. If you
    aren’t well-rested and are running on fumes, it’s a lot more difficult to be
    productive, stay focused, and motivate yourself.
  }
  \cue{Build a routine and healthy habits.}
  \note{
    Structure healthy habits like meals, sleep, exercise, and study times into
    your daily schedule and then stick with it. Motivating yourself to
    accomplish tasks becomes easier when you make it a part of your regular
    routine.
  }
  \cue{Eat and drink healthily.}
  \note{
    Drink enough water---your body needs water to function and improve energy.
    Eat regularly, don’t skip meals, and try to eat healthy foods. You need
    energy to complete tasks, and it’s much easier to get started and stay
    focused on work if you are well fed.
  }
\end{cuenotes}
%
\subsection{Metacognitive strategies}
Reflect on what makes you happy, what fulfills you, and what you are passionate
about. Try to align what you do with things that make you happy and fulfill you.
If you spend all of your time doing or pursuing things that you do not like or
care about, you may never be fully motivated. Choosing pathways and activities
that interest you is one of the biggest ways to better motivate yourself.

Give yourself rewards for accomplishing difficult tasks and identify strategies
that help keep you accountable.

Think about what support you need in order to achieve your goals and then get
the support you need. This could include investing in a new planner, attending
peer tutoring, or making an appointment with an academic coach at the Learning
Center.

Accept that you aren’t perfect. Many students lack motivation because they are
afraid of not performing as well as they would like. Combat your fear of failure
by telling yourself that your self-worth does not depend on your ability to
perform. Include your image of success to include personal and social success
and growth.

Write a letter to your future self to remind yourself of your goals. Read this
message when you find yourself feeling unmotivated.

Reflect. When you have a task to accomplish, reflect before, during, and after.
Think about your feelings towards the task, what you need to do to accomplish
it, and how you feel when you are done.

Talk to yourself out loud about your dreams and goals and speak encouraging,
positive words to yourself. Compliment yourself and tell yourself you can do it.

List out what is preventing you from doing what you need to do, then find ways
to tackle those things. Be specific.

Think long-term. Keep focused on your long-term goals and think about them when
you’re feeling unmotivated. Reminder yourself of how this task or step gets you
closer to your big goals. Print out a picture of where you want to be in the
future and post it on your wall or mirror.

If you’re feeling stuck, visualize yourself as you want to be in the future.
Picture yourself in your future career or situation and remind yourself of what
you are working for.

Stay positive and optimistic. Avoid complaining or commiserating at times when
you planned to make progress towards your goals. If the problems or obstacles
can be set aside till later, it may help to write them down to ensure you get
back to them. If there’s a problem that cannot be set aside, seek out resources
and support to help you address what’s wrong.

Think about consequences. Sometimes thinking about the negative consequences of
not doing a particular task you might be stuck on can be motivating.
Alternatively, think about the reward of accomplishing the goal (or at least the
feeling of being getting it over with) as a motivator.
%
\subsection{Accountability strategies}
Set visual reminders and alarms on your phone and laptop to remind and encourage
yourself of your goal. Consider changing the background of your phone and laptop
to a motivational quote or simply to saying the goal that you want to reach.
Create positive and encouraging visual reminders and motivators to hang on your
bedroom wall or mirror.

Share your goals with a friend, classmate, or someone in your life. Reach out to
someone and ask them to help keep you accountable with your work and goals.
Check in with this person face-to-face or online regularly to discuss your
progress.
%
\end{document}
