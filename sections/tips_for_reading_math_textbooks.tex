\documentclass[../main.tex]{subfiles}
\graphicspath{{\subfix{../figures/}}}
%
\title{Tips For Reading Math Textbooks}
%
\begin{document}
\maketitle
%
\section{Introduction}
Reading a math textbook is different than other types of reading.
A math textbook teaches you concepts and techniques rather than
telling you a story. It's not always the best strategy to start
from the beginning and try to read every single word.
If you feel like reading your math textbook is ``impossible''
try some of the following strategies, which will help you
get the most out of your reading.
%
\section{Tips}
\begin{cuenotes}
  \cue{Know your goal}
  \note{
    \begin{itemize}
      \item Figure out what concepts you should be getting
        from the reading---consult the syllabus,
        talk to your professor, or ask your TA.
      \item Are you reading in preparation for lecture?
        Focus mostly on getting the big picture and
        generating questions for class. Reading after lecture?
        Use the book to fill in details that
        you might have missed in class and work examples
        to get extra practice.
      \item Take a look at the types of problems you'll be expected to
        solve or the concepts you are
        expected to understand by the end of the section.
      \item Keep your goal in mind as you read,
        and check in with yourself after finishing a section.
    \end{itemize}
  }
  \cue{Take notes…in your own words}
  \note{
    \begin{itemize}
      \item When reading a math book, take notes as a way to
        translate the text into your own words.
        This is an effective learning technique---
        when you write down definitions, theorems
        and explanations in your own words,
        you are more likely to understand and remember them.
      \item If you're having trouble putting something
        in your own words, it might mean you don't
        have a solid understanding of it yet.
        Consult another resource(e.g. another textbook or
        internet resource) or person (e.g. professor, tutor, classmate).
        Even if the process is frustrating, it's worthwhile;
        testing your ability to put something in your own words
        is a good way to gauge what you know and don't know,
        which is an important part of the learning process.
    \end{itemize}
  }
  \cue{Work through examples}
  \note{
    \begin{itemize}
      \item Don't just skim (or worse, skip!) the examples.
        Instead, devote more of your focus to the
        examples. Working through them and
        verbalizing the main ideas behind them is a great
        way to test to make sure you're understanding what you're
        reading.
      \item Read with a pencil and paper in your hand and
        when you get to an example, work it out!
        Try not to look at the solution until you are done.
      \item When checking your work, make sure you understand each step and why.
      \item Practice the examples BEFORE you attempt homework problems and then try to do your
        homework without referring back to the example.
    \end{itemize}
  }
  \cue{Fill in the gaps}
  \note{
    \begin{itemize}
      \item Math textbooks, especially for higher-level classes,
        tend to leave out details they think are obvious.
        What might seem obvious to the textbook author is not obvious
        to everyone!
      \item If you are reading through a proof or a step-by-step solution to an example,
        and you can't see how the author got from one step to the next,
        try to fill in the missing details yourself.
        If you can't figure them out on your own,
        ask someone else (a classmate, peer tutor, TA, or professor).
    \end{itemize}
  }
  \cue{Try the problems at the end of the chapter}
  \note{
    \begin{itemize}
      \item Answering the questions and working through the problems
        at the end of the chapter is a great way to test your understanding.
        Too many students skip this great opportunity.
      \item Look at how the questions are organized---
        sometimes they are increase in difficulty or are arranged by topic.
      \item Make sure you try a diverse assortment of problems,
        not just all one topic or all of the easy ones.
    \end{itemize}
  }
  \cue{Re-visit the reading}
  \note{
    \begin{itemize}
      \item Have you ever seen a movie twice and understood so much more the second time?
        The same thing works for textbooks!
      \item Try reading through a section before you cover it in class to give yourself some context for
        the lecture. Focus mostly on ``big picture'' ideas and one or two examples.
      \item Return to what you read after class to help reinforce what you learned in and prepare for
        the homework assignment.
      \item You can always use your textbook as a reference when you forget a definition,
        theorem, or problem solving technique as well---just make sure to take
        the time to think through it yourself.
    \end{itemize}
  }
\end{cuenotes}
%
\end{document}
