\documentclass[../main.tex]{subfiles}
\graphicspath{{\subfix{../figures/}}}
%
\title{深度工作}
%
\begin{document}
\maketitle
\section{概念}
%
\begin{cuenotes}
  \cue{引子}
  \note{
    当专注变成了一种稀缺资源,分心和碎片化无时无刻都在蚕食我们专注于解决难题的能力。
    忙碌并不能真的转化成生产力,
    决定大多数知识工作者的核心竞争力的并不是忙碌,
    而是高质量的工作产出。
  }
  \note{
    你是否经常有这样的困扰:
    \begin{itemize}
      \item 在繁忙的多线程工作中不断地切换注意力,
        回邮件、开各种会议、查看并回复微信上几百上千的消息提示。
        经常被各种消息打断,无法专注。
      \item 无法抵挡即时通讯软件(微信)和各种社交媒体(抖音/微博/知乎)
        的诱惑,很难进入专注的工作状态。
      \item 明明感觉自己从早忙到晚,
        但肉眼可见地发现自己的产出效率很低,
        为了完成工作任务甚至被迫加班。
    \end{itemize}
  }
  \note{
    根据生产力公式 \emph{高质量工作产出$=$时间$\times$专注度},
    要提高个人的产出效率,需要长时间、无干扰地高度专注于单一任务,
    也就是\textbf{长时间的深度工作}。
  }
  \cue{什么是深度工作}
  \note{
    判断你的工作是否有深度,有两个概念:
    \begin{itemize}
      \item 深度工作(Deep Work):
        在无干扰的状态下专注进行职业活动,
        使个人的认知能力达到极限。这种努力能够创造新价值,
        提升技能,而且难以复制。
      \item 浮浅工作(Shallow Work):
        对认知要求不高的事物性任务,往往在受到干扰的情况下开展。
        此类工作通常不会为世界创造太多价值,且容易复制。
    \end{itemize}
  }
  \note{
    大部分情况下,深度工作需要高度集中注意力并应用复杂的知识和技能,
    也因此更能创造个人价值和产出,让你更具不可替代性。
    而浮浅工作虽然让你看起来很忙碌,但这种忙碌其实和生产力关联度不大。
    研究表明,知识工作者 60\% 以上的时间都花费在处理产出价值有限的浮浅事务上。
    因而增加深度工作的时间,
    降低浮浅工作在每天工作中的时间占比是提高你工作效率和核心竞争力的重要途径。
  }
  \cue{深度工作的意义}
  \note{
    \textbf{职业满足感}: \\
    谈到工作,很多人会觉得工作充满了压力、烦恼、沮丧和琐事。
    对于大部分人来说,工作只是谋生的工具,
    能从工作中获得成就感的人不多。
    但有些人却把工作当成一种可以培养个人技能并从中找到意义和价值的事情。
    编程奇才圣地亚哥·冈萨雷斯(Santiago Gonzales)
    在一次采访中这样描述自己的工作:
    \begin{quotation}
      精妙的代码简洁明了,如果将这个代码给其他程序员看,
      他们会说「哇,这代码写得真好。」那感觉很像在写一首诗。
    \end{quotation}
    享受深度专注所带来的快感,能让工作变成一件有满足感的事情。
    在深度工作中,不断提升自己的技能和能力,像手艺精湛的匠人一样,
    你可以在日常职业生活中创造出意义。
    \begin{itemize}
      \item 我写代码就像是在写诗,
        隔很久再修改我的代码的时候我都会觉得自己真是个人才。
      \item 我写的文章逻辑结构清晰,干货满满又兼具可读性。
      \item 我写的文档真是清晰又明确,开发完全没有理由 diss。
      \item 我认真肝出来的设计连我自己都觉得无比漂亮。
    \end{itemize}
  }
  \note{
    \textbf{最优体验(心流)}: \\
    ``一个人的身体或头脑在自觉努力完成某项艰难且有价值的工作过程中达到极限时,
    往往是最优体验发生的时候。''
    心理学家契克森米哈赖(Csikszentmihalyi)把这种心理状态称作``心流''。
    在全身心投入的工作中,将脑力开发到极限、专注,
    甚至达到忘我的境界。深度工作让我们更容易获得心流的体验。
    这种愉悦感,比瘫在沙发上刷抖音更让人享受其中。
    此外,全神贯注的状态,还可以让你更容易忽略生活中那些细小不快的事情,
    整个人变得更加积极。
  }
\end{cuenotes}
%
\section{如何做到深度工作}
\begin{cuenotes}
  \cue{将深度工作变成一种日常工作习惯}
  \note{
    培养深度工作的习惯,关键在于在工作生活中加入一些特别设计的惯例和固定程序,
    使得进入并保持高度专注状态消耗的意志力最小化。
  }
  \note{
    根据你自身的工作习惯,可以选择适合自己的\textbf{深度哲学}。 \\
    \textbf{双峰哲学}: \\
    双峰哲学将个人时间分成两块,将一块明确的时间用于深度追求,
    余下的时间用其他所有的事情。在深度时间内追求高强度、无干扰的专注,
    在肤浅工作时间处理零碎事情或者其他安排,灵活安排。
    这种做法可以在多个时间层级上实现:
    \begin{itemize}
      \item 按周划分,可以每周 4 天做深度工作,其余为开放时间;
      \item 按年划分,可以选一个季节完成大部分的深度工作
        (很多做学术的人都在夏季或休假期间完成);
      \item 按天划分,可以一天大部分时间做深度工作,
        每天抽一个小时集中回复邮件和处理消息。
    \end{itemize}
    \textbf{节奏哲学}: \\
    节奏哲学认为轻松启动深度工作的最好方法就是将其转化成一种简单的常规习惯,
    让自己不需要投入精力便知道让自己何时进入工作状态。
    比如早上六点起床一直工作到八点,然后出门去上班。
    关键是在于挤出固定时长无干扰的深度工作时间。
    节奏哲学的好处在于更符合人类的真实天性。
    节奏日程安排者通过雷打不动的习惯支持深度工作,
    确保能够定期完成一定的工作,在一年的时间里往往能够累积更多的深度工作时长。
    \\
    \textbf{禁欲主义哲学}: \\
    将自己放在一个相对封闭的环境中,从而避免干扰。
    选择这种模式的群体一般具有明确而且价值极高的职业目标追求,
    而且在职业上取得的大部分成就是由于工作表现突出,
    清晰的目标能够让他们排除纷杂的浮浅关注点。 \\
    \textbf{新闻记者哲学}: \\
    只要有空闲时间,立刻能进入深度工作模式。
    这种方式适合熟练的深度工作者,不适合新手。
    但如果你对自己深度工作对技能足够熟练,
    对自己从事的工作价值有足够的信心,
    这种方式可以帮你在紧密的日程安排中挤出大量的深度工作的时间。
  }
  \note{
    不管选择哪一种深度哲学的工作方式,很重要的一个点是要为自己设定深度工作的习惯。
    在哪里工作,工作时间有多长,工作开始后如何继续工作,
    如何支持自己的深度工作(散步或其他能让大脑保持清醒的活动)。
    逐渐培养属于自己的工作习惯,减少不必要的意志损耗,
    更快进入深度工作的状态。
    另外,为工作设定一个``完工仪式'',停止工作后不再想工作的事情。
    适当放松自己,安逸的时光有助于提升洞察力和补充深度工作所需的能量。
  }
  \cue{拥抱无聊,减少对分心事物的依赖}
  \note{
    研究发现,一旦你的大脑习惯了随时分心,即使你在想要专注的时候,
    也很难摆脱这种恶习。因而,在进行专注的训练时除了要做到高强度提高你集中注意力外,
    还要不断地去克服分心的欲望。要做到这两者,有一些可操作的建议。
  }
  \note{
    \textbf{计划好上网时间,其余时间避免使用网络}: \\
    通过分割网络使用(相当于分割了分心)来减少自己向分心屈服的次数,
    增强自己控制注意力的力量。不停地查看手机,
    回复微信阻碍了你的专注力。适时进行断网,可以从源头上切断干扰源。
    很短暂的干扰也会显著延长完成一项任务所需要的时间。当你在进行深度工作时,
    一个突然弹出的信息,你的大脑会对干扰做出反应,从而打断你工作的进程。
  }
  \note{
    \textbf{无聊时,将注意力集中在一件``定义明确的专业难题上''}: \\
    如果不减少对分心事务的依赖,增强专注度的努力可能就会白费。
    因而即使在无聊的时候,也不要把大脑交给肤浅的、
    容易获得及时反馈的社交、游戏、网页等。因为大脑一旦适应分心的状态,
    专注就变得异常困难。
    因而,即使在无聊的碎片化时间里(洗澡、散步、上班通勤路上),
    可以将注意力集中到一件定义明确的专业难题上,
    比如说一篇文章的提纲、一个商业计划等。这种有针对性的冥想,
    可以迅速提高你深度思考的能力。
  }
  \note{
    \textbf{给目标任务设定 deadline,限时完成}: \\
    根据最小阻力原则(The Principle of Least Resistance):
    在工作环境下,若各种行为对于底线的影响没有得到明确的反馈意见,
    我们倾向于采用当下最简单易行的行为。也就是说在缺乏度量的环境下,
    大多数人会选择最简单的工作。\\
    总统罗斯福的工作方式是首先找出一项优先性很高的深度任务,
    估算出完成此类型任务需要的时间,然后给自己设定一个 硬性截止期限,
    并且留出的时间远少于估算的时间。这就要求你利用每一束空闲的神经元来处理任务,
    直到你用自己不懈的高度精力来集中解决任务。
  }
  \cue{远离无效社交,谨慎选择网络工具}
  \note{
    社交媒体和即时通讯工具把我们的工作和生活分割成无数的碎片,
    削弱了我们集中注意力的能力。我们在上面白白消磨了很多时间和精力,
    甚至花费很多时间去和微博上持不同立场的人吵架。 \\
    关于``如何戒掉社交上瘾''这个问题,
    作者给出的答案是给自己的大脑找到更有意义的事情去做。
    可以给大脑找一些高质量的替代活动,比如说读一本有趣的小说,
    看一部有深度的电影,花两个小时进行画画,到户外去拍照等。
    这样不仅可以把我们从无效社交的黑洞中捞出来,还可以让我们避免分心,
    保持专注的能力。让我们体验到什么是生活,而不仅仅是生存。
  }
  \cue{摒弃浮浅,掌握工作的主动权}
  \note{
    作者并非完全否定浮浅工作,但过度的浮浅工作会占据你深度工作的时间。
    人的注意力是稀缺资源,当你被一些事情占据后,
    你就没有多余的时间和精力去做其他事情。作者提出了几点可操作性的建议。
  }
  \note{
    \textbf{减少浮浅工作的占比}: \\
    一天工作的 8 小时里,如果你的时间被各种会议、回复通讯消息、
    浏览网页等浮浅事务所占据,那么你集中注意花在深度工作上的时间势必会被挤压。
    作者举了 basecamp 的例子,basecamp 在 2007 年做了一个实验,
    把五天工作制缩减成四天,免去状态审查会和报告。工作时长虽然减少了,
    但员工的产出并没有受到影响。降低了浮浅工作的占比,
    让 basecamp 的员工更专注于重要的事情上。
  }
  \note{
    \textbf{做好工作日程规划}: \\
    为了避免被繁琐杂事占满日程,每天的工作事项都要提前做好规划。
    深度工作要求你尊重自己的时间,
    提前规划好你今天要做什么并思考如何让产出最大化。
    当然你的工作计划可以是灵活的,允许根据突发事件进行修改。
  }
  \cue{具体的方法建议}
  \note{
    根据作者四个深度工作的原则,我们可以总结出以下的方法建议。
    \begin{itemize}
      \item \textbf{养成习惯和流程}:
        将工作模块化,在固定时间集中处理无法避免的浮浅活动,
        如微信、回邮件等。
      \item \textbf{做好工作安排}:将一天的时间划分出两个不想被打扰的时间块,
        集中精力处理那些比较棘手的难题。
      \item \textbf{控制干扰源}:把手机 APP 的提醒功能关掉,
        比如微博,微信,淘宝,等各种社交网络。
      \item \textbf{做好工作复盘}:回顾每天/周的工作计划,
        定期对工作成果进行复盘。
      \item \textbf{适时断网}:在进行深度工作时,关闭即时通讯设备。
      \item 根据事项紧急程度确定是否要``秒回''信息。
        最好的方式是找个时间集中处理。
    \end{itemize}
  }
\end{cuenotes}

\summary{
  本文主要讲解如何更好地在无干扰的状态下工作,
  培养自己专注于解决难题的能力。
  深度的生活并不适合所有人。你需要为此付出艰苦的努力,
  从根本上改变你的习惯。对于很多人来说,
  快速地收发电子邮件和在社交媒体上发消息所带来的繁忙假象会给他们带来慰藉,
  深度的生活却要你摆脱这些东西。
  不管你如何抉择,祝你活出专注的人生。
}
%
\end{document}
