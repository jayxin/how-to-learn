\documentclass[../main.tex]{subfiles}
\graphicspath{{\subfix{../figures/}}}
%
\title{自学}
%
\begin{document}
\maketitle
%
\section{引子}
在焦虑感漫天飞舞的当代社会,保持自信笃定并不简单。
现实生活带来的挫败感无处不在,同辈压力如影随形。
快节奏的城市生活下,人越来越像旋转着的陀螺
---被惯性裹挟无法自主但也不敢停下。

各种技能速成班被开发出来,Python、原画、PPT、增长黑客、理财、
思维训练营…躲得过小时候的各种补习班,
躲不过朋友圈里各种``特效速成课程''。

斜杠青年、副业收入 、睡后收入等词花开遍地,
一瞬间教人如何赚钱成了非常赚钱的行当。
再加上``社畜''、``中年危机''、``阶级固化''、``消费主义''等词的反复撩拨,
年轻人很难不感到焦虑。

我们主观上渴望拥有更多技能,想点燃各种技能点并以此来应对这瞬息万变的商业社会。
然而被焦虑感驱动的功利性学习却往往达不到我们理想中的效果。

到底如何掌握我们所需的知识和技能?
今天和大家分享如何提高自学能力。
%
\section{学习的动力}
为什么学?自学的动力从何而来?

互联网如此发达的现在,知识的获取和学习,成本非常之低,
学习的主要障碍在于动力、惰性、以及压力源。

相比于被焦虑感驱使的被动学习,找到那个内在的驱动力更为重要。
自我驱动是主动选择的结果,主动地去进行选择和学习,
那么学习才会真正使你快乐,而不是痛苦了。 \\

\begin{cuenotes}
  \cue{打磨别人难以替代的技能(纵向深入)}
  \note{
    随着工作经历的增长,我们积累的行业技能会越来越多。
    但工作中其实不是``木桶理论'',而是最长的板决定你不可替代的优势。
    因而,打造局部压倒性优势显得尤为重要。
  }
  \note{
    不断强化自己的突出优势,打磨自己的核心竞争力。
    在自己的领域内,与时俱进,不断去探索新的发展和可能。
    Do what you love, love what you do.
    如果你已在自己喜欢且擅长的领域内,那么已是非常幸运。
  }
  \cue{多学科整合的叠加效应(横向发展)}
  \note{
    很多岗位需要综合各种能力,比如运营需要数据分析的能力、
    增长和运营的能力、文案和活动策划能力、沟通和推进项目的能力…
    但跳脱出这个常规的能力框架范围,去做一些其他领域的积累,
    会有意想不到的效果。
  }
  \note{
    比如说设计师懂一点编程会对整体的设计有帮助,
    产品懂一点代码的实现方式和心理学会对功能设计有帮助,
    开发英语好对搜索资料和解决问题很有帮助。
  }
  \cue{为解决问题驱动的学习}
  \note{
    ``在用中学''是一个非常高效的学习方式。
    根据情境学习(Situated Learning)理论,在要学习的知识、
    技能的应用情境中进行学习,因其自带的极强应用目的,
    能有效驱使你自发地进行学习,并``学以致用''。
  }
  \note{
    比如工作中遇到难题而驱动的自我学习,
    这个学习场景在程序员编程过程中非常常见。
    学了立即能应用到实践中,这种自驱力是特别强的。
  }
  \cue{工作之余的其他发展}
  \note{
    如果你的才能和特长在平时的工作中没有办法得到充分的发挥,
    可以尝试着去做工作内容外的探索。其实各种厉害的网红、
    Up 主和 Vlog 博主们也是起于平地。
    抛弃打工者的心态,以增长个人能力目的驱动的学习会收获更多满足感。
  }
  \cue{纯兴趣驱动的学习}
  \note{
    有一个甚至可以多个被称为``骨灰级爱好''的东西其实非常幸福。
    不管是画画,还是书法;不管是跳舞,还是瑜伽;阅读、电影、哲学、
    乐器、健身等.
    相比于沉浸在各种娱乐八卦的信息流中,
    把时间花在喜欢的事情上更容易产生专注的快感,
    也更容易让人放松。
  }
\end{cuenotes}

弄清楚为什么学很重要,在能力范围内去做有意识的思考,
去找到适合自身发展的发力点。因为主动学习的效率远比盲目被动地进行知识的填鸭来得高效。
而当你找到那个势不可挡的自驱力时,也大半知道自己要学什么了。
%
\section{学习的内容}
找到学习的动力后,接下来我们从理论知识和实践技能这两个方面来聊聊如何更高效地学习。

\subsection{系统化地学习一门学科(理论知识)}
我们平常接触到的公众号推文、新闻、博客文章、播客等
其实都是碎片化的信息。而知识体系是高度结构化的、相互关联的、
纵横交错的知识网络。

信息$ \rightarrow $知识点$ \rightarrow $结构化的知识$ \rightarrow $知识体系.

系统化学习的要点在于构建知识框架,在于把知识点按照一定的逻辑顺序进行分类和编排形成知识结构,
众多结构化的知识,交叉编排,互相关联从而形成知识体系。
体系化的知识更容易产生连结,也更容易提取和应用。

具体做法有:
\begin{itemize}
  \item 找到好的系统性入门书
  \item 找到相关的入门课程
\end{itemize}
%
\subsection{习得一项技能/兴趣/爱好(实践技能)}
正如前面我们提到的``在用中学''是一个非常高效的学习方式。
当你以解决实际问题为目的出发去学习一项技能时,驱动力会更强。
如何更高效地习得一项技能?我们可以:
\\

\begin{cuenotes}
  \cue{善于利用各专业论坛、社区、学习平台}
  \note{
    \begin{itemize}
      \item Coursera
      \item B 站
      \item 知乎
      \item GitHub
      \item 各在线学习平台
      \item etc.
    \end{itemize}
  }
  \note{
    以 Coursera 为例,比如你想学习设计基础,
    可以去 Coursera 上去搜索课程,根据自己想学的主题和水平去筛选。
    上面多数课程是免费的,也会有介绍明确你可以学到什么。
    课程介绍让你更好的了解课程的各个阶段,
    学生的评论可以帮助你确定他是不是符合你的预期。
    Coursera 的好处在于结合了理论讲解和作业的实际操作。
    此外,如果你想获得证书为你的技能点加分,
    Coursera 可以直接 link 到领英。
  }
  \cue{善于利用搜索引擎}
  \note{
    拥有极强解决问题能力的人大概率也是一个善于搜索的人。
    如何更高效地进行搜索,掌握一定的搜索技能至关重要,分享几个 tips:
  }
  \note{
    \begin{itemize}
      \item 把关键词放入双引号中,进行完全匹配搜索,
        比如:\texttt{"系统性入门书法"}
      \item 用 filetype 搜索特定文件格式,
        比如:\texttt{filetype:pdf 《刻意练习》}
      \item 站内搜索,
        比如:\texttt{site:zhihu.com 思维导图}
      \item 利用通配符 `*' 进行模糊搜索,
        比如查询诗的下一句:\texttt{海底月是天上月*}
      \item 减号排除,缩小范围,
        比如搜索不包括课程的``数据分析''搜索结果:
        \texttt{数据分析-课程}
      \item 选取适合的搜索关键词,如果一个关键词搜索不到,
        可以采取相近的关键词或进行关键词的组合
    \end{itemize}
  }
  \cue{学会提问,给自己找好老师}
  \note{
    如果你对学习的具体内容还不是很明确,比如做一个网页。
    这个里面包含的所需的技能会比较多,
    可以先去知乎或技术论坛上先查看相关的问题,
    明确自己需要学习哪些内容。一般知乎上面会有推荐的阅读材料,
    以及课程(见仁见智)。
  }
  \note{
    在学习的过程中,遇到不明白的地方可以寻求程序员朋友的帮助。
    学会提问是一项技术,在发问前,善于利用搜索引擎,
    比用\texttt{site:stackoverflow.com}求助,实在不大明白的话,
    再咨询朋友会更合适。
  }
  \cue{Just Do It}
  \note{
    很多实操性强的工具有时候并不需要那么多的理论指导,
    比如学会使用XMind。XMind 是一款易用性非常强的软件,
    直接上官网下载安装,并学会几个基础的操作后即可入门。
  }
\end{cuenotes}

综上,总结下如何更高效地获取优质的学习素材:
\begin{itemize}
  \item \textbf{专题阅读}:阅读这个知识领域的经典教材、经典书籍
  \item \textbf{公开课、付费课程}:体系化授课,能让你更系统地进行学习
  \item \textbf{专业大牛}:请教行业领域的牛人,让他们推荐书籍,带你入门
  \item \textbf{专业网站、论坛}:相关的专业网站(GitHub等)、
    论文库、知乎、Quora、Wikipedia、Wikibooks 或其它专业类网站
  \item \textbf{高质量媒体、博客}:知名媒体、个人博客、细分领域公众号、
    高质量分享会等
  \item \textbf{强大的搜索引擎}
\end{itemize}
%
\section{如何更高效地学习}
在找到优质的学习素材后,简单聊聊如何更高效地进行学习。
\\

\begin{cuenotes}
  \cue{费曼原理}
  \note{
    费曼技巧是一种``以教为学''的学习方式,可以帮你提高知识的吸收效率,
    真正理解并学会运用知识。这个学习方法其实很简单,
    就是验证你是否真正掌握一个知识,
    看你能否用直白浅显的语言把复杂深奥的问题和知识讲清楚。
    当你学习完后,可以进行文章的产出或者和其他人分享你的学习成果,
    不断强化理解和复述直到真正掌握。
  }
  \cue{叠加效应}
  \note{
    当你想学英语时,可以把学英语当作手段,而不是目的。
    比如,可以找寻英语语境中所在领域的内容,进行知识的积累。
    比如上 Coursera 学习感兴趣的国外的相关课程,
    看感兴趣的英文的专业书籍,写英文的技术 blog,
    在英文环境中分享自己的领悟和经验等。
  }
  \cue{刻意练习}
  \note{
    对技能的习得,进行针对性的刻意练习和深度学习。
    ``刻意练习''的本质是长时工作记忆。
    比如在进行钢琴、象棋等专业活动时,能够调用更大容量的工作记忆。
    通俗来说,就是他们拥有更大的内存。
    这种长时工作记忆可以通过一定的练习来进行激活,通过一定难度的重复练习,
    在每次练习中收到反馈,不断纠正自己的错误,
    以获取提升。书法、健身、绘画等实操性技能可以通过这种方法来进行提高。
  }
  \cue{积累式学习}
  \note{
    对一切保持好奇心和探索欲,遇到感兴趣的知识点及时进行归纳和总结。
    不断扩充和累积自己的知识面,不管这个知识是否有用。
    比如当你某天对平常喝的茶感到好奇,
    于是你搜索了茶叶的种类并顺手进行了整理。从此你便知道了茶叶有哪些,
    也知道了有哪些知名的品种。
  }
  \cue{问题驱动式}
  \note{
    通过解决问题学习
  }
\end{cuenotes}

好奇心和求知欲带我们走得更远。主动学习不仅能有效地消除焦虑感,
在学的过程中我们也能收获快乐和满足。

每个人都有自己的学习方法,或许以上方法并不一定适合每个人,
但有一点准没错的是,在你学习的过程中,
用 XMind 进行知识的梳理和总结能有效提高学习的效率,让你记得更牢,
学得更轻松。

%\summary{}
%
\end{document}
