\documentclass[../main.tex]{subfiles}
\graphicspath{{\subfix{../figures/}}}
%
\title{Procrastination}
%
\begin{document}
\maketitle
%
\section{Introduction}
Everyone procrastinates, but that does not mean it's inevitable.

You can stop procrastinating today. But you might need some help understanding
why you do it and how you can stop. Here, you can learn why procrastination
happens, find some easy tips to stop procrastinating now, and arm yourself with
useful anti-procrastination tools that you can use on campus or at home.
%
\section{Why we procrastinate}
\begin{cuenotes}
  \cue{Because we're wired to seek instant gratification.}
  \note{
    Chances are good that you have Facebook, Instagram, and/or Twitter pulled up
    in another window on the device you're using to read these very words. And
    it's so tempting to keep checking it, right? There's a reason for that:
    research suggests that instant gratification has a stronger effect on our
    behavior than delayed gratification.
  }
  \cue{Because we think we should be perfect.}
  \note{
    Procrastination and perfectionism often go hand in hand. Perfectionists tend
    to procrastinate because they expect so much of themselves, and they are
    scared about whether they can meet those high standards. Perfectionists
    sometimes give a half-hearted effort in order to maintain the belief that
    they could have written a great paper if only they had tried their best.
    They are afraid of trying their best and still producing a paper that is
    just okay.
  }
  \cue{Because we don't like what we need to do.}
  \note{
    You may procrastinate on writing because you don't like to re-read what you
    have written; you hate writing a first draft and then being forced to
    evaluate it. By procrastinating, you ensure that you don't have time to read
    over your work, thus avoiding that uncomfortable moment.
  }
  \cue{Because we're too busy.}
  \note{
    When we overbook our calendars, it's easy to avoid the things we don't want
    to do, even if we need to do them.
  }
\end{cuenotes}
%
\section{How to tame procrastination}
\begin{cuenotes}
  \cue{Take an inventory}
  \note{
    Keeping track of when you procrastinate with a weekly planner can help you
    figure out when you procrastinate and can help you stop the behavior. It's
    easy to do: whenever you procrastinate, mark it down. Think about clues that
    can alert you: for example, a nagging voice in your head, a visual image of
    what you are avoiding or the consequences of not doing it, physical ailments
    (stomach tightness, headaches, muscle tension), inability to concentrate, or
    inability to enjoy what you are doing.
  }
  \cue{Create a productive environment}
  \note{
    If you've made the decision to stop delaying on a particular project, it's
    critical to find a place to work where you have a chance of getting
    something done. Your dorm room or your bedroom may not be the place where
    you're most productive. Think about where you are most productively and try
    to find or create a space with those conditions. If you're working at home,
    try finding a space that you devote solely to studying, like a desk or a
    comfy chair. Make sure to find your study space before it's time to be
    productive; otherwise, finding the perfect space could turn into a form of
    procrastination itself!
  }
  \note{
    One useful way to structure your environment is to leave yourself reminders
    to work in places you know you'll see (like your bathroom mirror or coffee
    machine). Once you're in a productive space, eliminate digital distractions.
    Pull up the materials you need on your laptop, and turn the Wi-Fi off and
    put your phone on airplane mode.
  }
  \cue{Challenge your myths}
  \note{
    Think of a project that you are currently putting off. On one side of a
    piece of paper, write down all the reasons for your delay. On the other
    side, argue against the delay.
  }
  \note{
    \begin{itemize}
      \item \textbf{Myth \#1}: ``I can't function in a messy environment. I can't
        possibly work on this project until I have cleaned my apartment.''
      \item \textbf{Challenge}: If, when faced with a project, you start piling
        up prerequisites for all the things you must do before you can possibly
        start working, consider whether you might in fact be making excuses—in
        other words, procrastinating.
      \item \textbf{Myth \#2}: ``I do my best work under pressure.''
      \item \textbf{Challenge}: There are other ways to create pressure for
        yourself besides waiting until the night before the project is due
        before you start working on it. You can set a time limit for
        yourself—for example, ``I will write this paragraph in half an hour''–or
        you can pretend that the assignment is a timed exam. If you do this a
        week or two before the assignment is due, you'll have a draft in plenty
        of time to revise and edit it.
      \item \textbf{Myth \#3}: ``In order to be productive, I must have two uninterrupted
        hours.''
      \item \textbf{Challenge}: You can work on assignments in one hour blocks
        (or shorter), and many people benefit from working in shorter blocks.
        This will help you break the task down into smaller pieces, thereby
        making it seem more manageable. If you know that you can work on one
        part of the project for one hour, then it won't seem so daunting, and
        you will be less likely to procrastinate. Some people find, however,
        that they do need longer blocks of time in order to really produce
        anything. Therefore, like all of the strategies outlined here, know
        yourself.
    \end{itemize}
  }
  \cue{Break it down}
  \note{
    The day you get an assignment, break it up into the smallest possible
    chunks. Using the Learning Center's weekly action plan can help. When you
    break a project down, it never has a chance to take on gargantuan
    proportions in your mind. If you're working on a research paper, for
    example, you can say to yourself, ``Right now, I'm going to write the
    introduction. That's all, just the introduction!'' And you may be more
    likely to sit down and do that, than you will to sit down and ``write the
    paper.'' If you're working remotely and feel overwhelmed by assignments for
    multiple classes, breaking your assignments down into smaller tasks can help
    your week feel more manageable. Focusing on studying one chapter of your
    chemistry textbook and writing an introduction to your research paper and
    building from there may feel more manageable than focusing on having a
    chemistry test and a research paper due in the same week.
  }
  \cue{Ask for help}
  \note{
    Get an anti-procrastination buddy. Tell someone about your work goal and
    timeline, and ask them to help you determine whether or not your plan is
    realistic. You can do the same for him or her. Once or twice a week, email
    your buddy to report on your progress, and declare your promise for the next
    week. If, despite your good intentions, you start procrastinating again,
    don't think, ``All is lost!'' Instead, talk to your buddy about it. They may
    be able to help you put your slip into perspective and get back on track.
  }
  \cue{Learn how to tell time}
  \note{
    One of the best ways to combat procrastination is to develop a more
    realistic understanding of time. Our views of time tend to be fairly
    unrealistic. ``This paper is only going to take me about five hours to
    write,'' you think. ``Therefore, I don't need to start on it until the night
    before.'' What you may be forgetting, however, is that our time is often
    filled with more activities than we realize. On the night in question, for
    instance, let's say you go to the gym at 4:45 PM. You work out (1 hour),
    take a shower and dress (30 minutes), eat dinner (45 minutes), and go to a
    sorority meeting (1 hour). By the time you get back to your dorm room to
    begin work on the paper, it is already 8:00 PM. But now you need to check
    your email and return a couple of phone calls. It's 8:30 PM. before you
    finally sit down to write the paper. If the paper does indeed take five
    hours to write, you will be up until 1:30 in the morning—and that doesn't
    include the time that you may spend watching Netflix or scrolling through
    Instagram. And, as it turns out, it takes about five hours to write a first
    draft of the essay. You have forgotten to allow time for revision, editing,
    and proofreading. You get the paper done and turn it in the next morning.
    But you know it isn't your best work, and you are pretty tired from the late
    night, and so you make yourself a promise: ``Next time, I'll start early!''
  }
  \cue{Make an unschedule}
  \note{
    The next time you have a deadline, try using an unschedule to outline a
    realistic plan for when you'll work. An unschedule is a weekly calendar of
    all the ways your time is already accounted for, so you include not only
    classes but also activities such as meals, exercise, errands, laundry, and
    socializing. This will give you an outline of the time that you spend doing
    other things besides studying.
  }
  \note{
    An unschedule will reveal your blank spaces: these are the times to schedule
    work. By using these as a guide, you'll be able to more accurately predict
    how much time you can study on any given day.
  }
  \note{
    An unschedule might also be a good way to get started on a larger project
    such as a term paper or an honors thesis. You may think that you have ``all
    semester'' to get the writing done, but if you really sit down and map out
    how much time you have available to work on a daily and weekly basis, you
    will see that you need to get started sooner, rather than later.
  }
  \note{
    Perhaps most importantly, an unschedule can help you see how you spend your
    time. You may be surprised at how much (or how little) time you spend on
    social media and decide to make a change. It's especially important that you
    build time for fun activities into your unschedule. Otherwise, you might
    procrastinate because you need time for relaxation.
  }
  \note{
    You can also use the unschedule to record your progress towards your goal.
    Each time you work on a paper, for example, mark it on the unschedule. One
    of the most important things you can do to kick the procrastination habit is
    to reward yourself when you write something, even if that writing is only a
    little piece of the whole. Seeing your success recorded will help reinforce
    the productive behavior, and you will feel more motivated to write later in
    the day or week.
  }
  \cue{Set a time limit}
  \note{
    Okay, so maybe one of the reasons you procrastinate on working on a
    particular assignment is because you hate it! You would rather be at the
    dentist than sitting in front of your desk with this problem set staring you
    in the face. In that case, it may be helpful to set limits on how much time
    you will spend working on it before you do something else. While the
    notation ``Must work on Hemingway essay all weekend'' may not inspire you to
    sit down and write, ``Worked on Hemingway essay for ½ hour'' just might.
  }
  \note{
    A lot of students find the Pomodoro Technique a helpful way to build in
    breaks. Pomodoro technique is simple: All you do is set a timer for 25
    minutes, work during that time, and take a 5-minute break when the timer
    goes off. Rinse, repeat. It's a great way of avoiding burnout!
  }
  \cue{Practice self-forgiveness}
  \note{
    Research suggests that forgiving yourself for procrastinating in the past
    can help you procrastinate less in the future. It's a way of acknowledging
    that procrastination is something you can change. Remember: The past is in
    the past. Let it go.
  }
  \cue{Take a social media hiatus}
  \note{
    There are only so many cat pictures you can look at before social media
    becomes counterproductive. It can be detrimental to your GPA in two ways:
    \begin{itemize}
      \item By taking up your time: Research suggests that hours spent on
        Facebook are negatively correlated to GPA.
      \item By taking up your attention: Yet more research suggests that texting
        while studying interferes with your mental bandwidth and ability to
        deeply learn material.
    \end{itemize}
  }
  \note{
    What is to be done? You can take a social media hiatus. It's simple: You can
    start by swearing off social media for two hours. You'll be amazed at how
    many times you automatically move to check Twitter or Instagram. See if you
    can gradually build your endurance: Can you stay off social media for four
    hours? A day? A week? If you tame social media, you'll have loads more time
    to work, play, and sleep.
  }
\end{cuenotes}
%
\section{How can technology help}
Technology can help intervene at various stages to help you prevent procrastination:
\\

\begin{cuenotes}
  \cue{Enforce your social media hiatus.}
  \note{
    Use distraction-blocking applications like StayFocused, SelfControl, and
    Serene that allow you to blacklist distracting websites on your desktop or
    smartphone. Many of these apps also integrate short timers that encourage
    you to work in short, manageable sessions.
  }
  \cue{Set a timer.}
  \note{
    Many smartphones and computers allow you to set a timer that will help you
    set a time limit for tasks. Timer apps such as MultiTimer and Goodtime
    include advanced features for switching between scheduled work and break
    periods.
  }
  \cue{Calendars for creating an unschedule.}
  \note{
    Using an online calendar like Google Calendar or iCal to create events that
    recur monthly, weekly, or daily can help you see your schedule from a bird’s
    eye view and identify where you have time to complete a task. You can also
    set reminders to signal when it’s time to get started on a task.
  }
  \cue{Checklists and sticky notes for breaking down projects.}
  \note{
    Make a checklist to break down a large project into smaller, more manageable
    tasks. Write out your checklist by hand, or create a quick qlist.cc online.
    Then put your current task on a sticky note to create a visual reminder—or
    add a virtual sticky note to your MacOS or Windows desktop.
  }
\end{cuenotes}
%
\section{Parting thoughts}
As you explore why you procrastinate and experiment with strategies for working
differently, don’t expect overnight transformation. You developed the
procrastination habit over a long period of time; you aren’t likely going to
break it all at once. But you can change the behavior, bit by bit. If you stop
punishing yourself when you procrastinate and start rewarding yourself for your
small successes, you will eventually develop new writing habits. And you will
get a lot more sleep.

In addition to these tips, check out some of our other handouts and resources to
help you with procrastination, such as our handouts on motivation, distractions,
and digital distractions.
%
\end{document}
