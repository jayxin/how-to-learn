
\documentclass[../main.tex]{subfiles}
\graphicspath{{\subfix{../figures/}}}
%
\title{Meta-cognitive Study Strategies}
%
\begin{document}
\maketitle
%
\section{Introduction}
Do you spend a lot of time studying but feel like your hard work doesn't help
your performance on exams? You may not realize that your study techniques, which
may have worked in high school, don't necessarily translate to how you're
expected to learn in college. But don't worry—we'll show you how to analyze your
current strategies, see what's working and what isn't, and come up with new,
more effective study techniques. To do this, we'll introduce you to the idea of
“metacognition,” tell you why metacognition helps you learn better, and
introduce some strategies for incorporating metacognition into your studying.
%
\section{What is metacognition and why should I care?}
Metacognition is thinking about how you think and learn. The key to
metacognition is asking yourself self-reflective questions, which are powerful
because they allow us to take inventory of where we currently are (thinking
about what we already know), how we learn (what is working and what is not), and
where we want to be (accurately gauging if we've mastered the material).
Metacognition helps you to be a self-aware problem solver and take control of
your learning.

By using metacognition when you study, you can be strategic about your approach.
You will be able to take stock of what you already know, what you need to work
on, and how best to approach learning new material.
%
\section{Strategies for using metacognition when you study}
Below are some ideas for how to engage in metacognition when you are studying.
Think about which of these resonate with you and plan to incorporate them into
your study routine on a regular basis.
\\

\begin{cuenotes}
  \cue{Use your syllabus as a roadmap}
  \note{
    Look at your syllabus. Your professor probably included a course schedule,
    reading list, learning objectives or something similar to give you a sense
    of how the course is structured. Use this as your roadmap for the course.
    For example, for a reading-based course, think about why your professor
    might have assigned the readings in this particular order. How do they
    connect? What are the key themes that you notice? What prior knowledge do
    you have that could inform your reading of this new material? You can do
    this at multiple points throughout the semester, as you gain additional
    knowledge that you can piece together.
  }
  \cue{Summon your prior knowledge}
  \note{
    Before you read your textbook or attend a lecture, look at the topic that is
    covered and ask yourself what you know about it already. What questions do
    you have? What do you hope to learn? Answering these questions will give
    context to what you are learning and help you start building a framework for
    new knowledge. It may also help you engage more deeply with the material.
  }
  \cue{Think aloud}
  \note{
    Talk through your material. You can talk to your classmates, your friends, a
    tutor, or even a pet. Just verbalizing your thoughts can help you make more
    sense of the material and internalize it more deeply. Talking aloud is a
    great way to test yourself on how well you really know the material. In
    courses that require problem solving, explaining the steps aloud will ensure
    you really understand them and expose any gaps in knowledge that you might
    have. Ask yourself questions about what you are doing and why.
  }
  \cue{Ask yourself questions}
  \note{
    Asking self-reflective questions is key to metacognition. Take the time to
    be introspective and honest with yourself about your comprehension. Below
    are some suggestions for metacognitive questions you can ask yourself.
  }
  \note{
    \begin{itemize}
      \item Does this answer make sense given the information provided?
      \item What strategy did I use to solve this problem that was helpful?
      \item How does this information conflict with my prior understanding?
      \item How does this information relate to what we learned last week?
      \item What questions will I ask myself next time I'm working these types
        of problems?
      \item What is confusing about this topic?
      \item What are the relationships between these two concepts?
      \item What conclusions can I make?
    \end{itemize}
    Try brainstorming some of your own questions as well.
  }
  \cue{Use writing}
  \note{
    Writing can help you organize your thoughts and assess what you know. Just
    like thinking aloud, writing can help you identify what you do and don't
    know, and how you are thinking about the concepts that you're learning.
    Write out what you know and what questions you have about the learning
    objectives for each topic you are learning.
  }
  \cue{Organize your thoughts}
  \note{
    Using concept maps or graphic organizers is another great way to visualize
    material and see the connections between the various concepts you are
    learning. Creating your concept map from memory is also a great study
    strategy because it is a form of self-testing.
  }
  \cue{Take notes from memory}
  \note{
    Many students take notes as they are reading. Often this can turn notetaking
    into a passive activity, since it can be easy to fall into just copying
    directly from the book without thinking about the material and putting your
    notes in your own words. Instead, try reading short sections at a time and
    pausing periodically to summarize what you read from memory. This technique
    ensures that you are actively engaging with the material as you are reading
    and taking notes, and it helps you better gauge how much you're actually
    remembering from what you read; it also engages your recall, which makes it
    more likely you'll be able to remember and understand the material when
    you're done.
  }
  \cue{Review your exams}
  \note{
    Reviewing an exam that you've recently taken is a great time to use
    metacognition. Look at what you knew and what you missed. Try using this
    handout to analyze your preparation for the exam and track the items you
    missed, along with the reasons that you missed them. Then take the time to
    fill in the areas you still have gaps and make a plan for how you might
    change your preparation next time.
  }
  \cue{Take a timeout}
  \note{
    When you're learning, it's important to periodically take a time out to make
    sure you're engaging in metacognitive strategies. We often can get so
    absorbed in “doing” that we don't always think about the why behind what we
    are doing. For example, if you are working through a math problem, it's
    helpful to pause as you go and think about why you are doing each step, and
    how you knew that it followed from the previous step. Throughout the
    semester, you should continue to take timeouts before, during or after
    assignments to see how what you're doing relates to the course as a whole
    and to the learning objectives that your professor has set.
  }
  \cue{Test yourself}
  \note{
    You don't want your exam to be the first time you accurately assess how well
    you know the material. Self-testing should be an integral part of your study
    sessions so that have a clear understanding of what you do and don't know.
    Many of the methods described are about self-testing (e.g., thinking aloud,
    using writing, taking notes from memory) because they help you discern what
    you do and don't actually know. Other common methods include practice tests
    and flash cards—anything that asks you to summon your knowledge and check if
    it's correct.
  }
  \cue{Figure out how you learn}
  \note{
    It is important to figure out what learning strategies work best for you. It
    will probably vary depending on what type of material you are trying to
    learn (e.g. chemistry vs. history), but it will be helpful to be open to
    trying new things and paying attention to what is effective for you. If
    flash cards never help you, stop using them and try something else instead.
  }
\end{cuenotes}
%
\end{document}
