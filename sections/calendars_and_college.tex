\documentclass[../main.tex]{subfiles}
\graphicspath{{\subfix{../figures/}}}
%
\title{Calendars And College}
%
\begin{document}
\maketitle
%
\section{Introduction}
This handout discusses why you need a calendar, shares strategies for managing
your time, and connects you with resources to help you make the most of each day
in college.
%
\section{Why you need a calendar}
Between rigorous classes, assignments and studying, new friends and social
events, obligations at home, and extracurricular obligations, managing time in
college can be a challenge. You may have more freedom, flexibility, and
independence, which can lead to overscheduling, procrastinating, or falling
behind in classes.

However, managing your schedule in college doesn’t have to be overwhelming; with
a little planning and structure, you can be better equipped to own your time,
live a balanced life, and stay on top of your work. And it pays off: studies
have shown that students who plan their time were more efficient in allocating
their individual study time, prepared more appropriately for the tutorial group
meeting, and achieved higher scores on cognitive tests (Van den Hurk, 2006).

This handout shares several types of planning that you can do to manage your
time and connects you with resources to help you make the most out of each day.
Note that you can use an online calendar (e.g., Google calendar, iCalendar,
Outlook, etc.) for semester planning and weekly planning as well.
%
\section{Semester planning}
Whether you’re using a wall calendar, a hard-copy planner, or an online
calendar, it’s good to have a way to keep an eye on the big picture for your
semester.

Here are some steps to effectively plan for your semester using a calendar:
\begin{itemize}
  \item Review your syllabi and enter the dates of all exams, papers, projects,
    events, and travel into your planner or semester-at-a-glance calendar.
  \item Work backwards from each due date or exam to plan when you want to start
    working on each task. For example, for papers, you may want to map out when
    you’ll do research, when you’ll start your draft, when you want to finish
    your first draft, and when you want to take your draft to the Writing
    Center.
  \item Add new appointments as they arise! These may include doctor’s
    appointments, meetings on campus, office hours you plan to attend, or social
    occasions you don’t want to miss. Set a reminder on your phone if it would
    help you remember your appointments.
\end{itemize}
%
\section{Weekly planning}
Students generally benefit from regularly setting aside time to think ahead and
plan for the week ahead. This kind of planning helps you make sure you allocate
enough time for each of your courses and helps avoid unforeseen pile-ups of
work.

Here are some steps to effectively plan for the week:
\begin{itemize}
  \item Have a regular time each week (budget 15–30 mins) to look at your
    assignments and obligations and map them out over the week.
  \item Mark and label time slots occupied by classes, employment, sports,
    extracurricular activities, chores, and other regular commitments. If you’re
    using an electronic calendar, it’s easy to make these occurrences recurring
    events over many weeks or months.
  \item Consider which activities have time restraints. For example, if you need
    to use the BeAM makerspaces on UNC’s campus, make sure you account for their
    hours. If your internet has low bandwidth, consider coordinating when you
    use the internet or stream lectures with your roommates or your family.
  \item For each class, create a weekly to-do list, estimating how much time to
    allot for each reading, assignment, paper, project, and study prep. Insert
    these tasks into open slots on your weekly planner, building in buffer time
    (don’t forget to eat!).
  \item Use color to differentiate classes and activities visually.
  \item You may also want to set aside a shorter amount of daily time to make
    your plan or to-do list for the day.
\end{itemize}
%
\section{Tips to make your weekly planning work}
Reduce your tasks to manageable steps or segments, rather than doing assignments
in long sessions. Work backwards from a target date and distribute the load
across your schedule. Example: Rather than reading 50 pages of a dense textbook
in one sitting, try 10 pages at a time.

Know when and where you work best. Plan for your hardest studying when you know
you’re most alert and focused. Consider your options and select the study
location where you can stay productive.

Maximize breaks in your schedule. Set a timer to remind you to take a quick
break every hour. Incorporate movement or exercise into your breaks to help you
stay focused during study time.

If possible, arrive to class early to review and stay afterward to clarify
lecture material while it’s fresh. If there isn’t time before or after class,
record your questions and take them to your instructor during office hours. If
you are taking online classes, settle in a few minutes before class starts to
review your notes. Use small breaks and down time to incorporate study. For
example, review flashcards while standing in line, heating up your lunch, or
riding the bus.

Strive for balance. Don’t just assume you’ll find time to take care of yourself
– schedule time for self-care (exercise, meditation, etc.) and even for free
time if you want it.

Adjust your schedule as needed. Unexpected events can interrupt even the
best-made plans. Flexibility and creative thinking prevent unforeseen
circumstances from derailing your daily or weekly goals. If your schedule gets
thrown off, readjust and keep going as best you can. Regular interaction with
your calendar is important in helping you be realistic and in helping you get
into the habit of using a calendar.

Set a stopping time at night. Sleep deprivation affects attention, cognition,
and memory. Mark your goal bedtime on your calendar each night and try to stick
to that time. Similarly, set a time to wake up every morning, mark that on your
calendar, and try to stick to it.

Keep your planner with you. If you’re using a hard-copy planner, it’s important
to have it with you at most times so you’ll have it when you need it. Refer to
it and update as needed. Electronic calendars that are on your phone make this
easy.
%
\end{document}
