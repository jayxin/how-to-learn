\documentclass[../main.tex]{subfiles}
\graphicspath{{\subfix{../figures/}}}
%
\title{Reading Comprehension Tips}
%
\begin{document}
\maketitle
%
\section{Introduction}
Do you ever feel overwhelmed with the amount of reading you have? Do you ever
have trouble staying focused and motivated while reading? Do you sometimes have
difficulty understanding and remembering what you read? If so, you're not alone.
Many students struggle with these things because reading in college can be
challenging, time-consuming, and lot more rigorous than high school; however,
with some effective strategies, you can make your reading time meaningful,
focused, and productive.
%
\section{Active reading}
Research shows that you retain more when you actively engage and interact with
texts, as opposed to simply reading and re-reading without a clear purpose. Many
students can relate to the type of reading that involves copying down pages of
notes word-for-word from the text or simply scanning over pages without really
reading them or interacting at all. While these two approaches are on opposite
ends of the spectrum, neither of them engages your brain in a way that elicits
deep understanding and retention. Active reading engages your brain in effective
strategies that force your brain to interact with the text before, during, and
after reading and that help you better gauge what you are (and are not)
learning.
%
\section{Before reading}
Although many students don't think about this step, engaging with a text before
reading can crucially boost your understanding and retention. Below are some
active reading strategies to use before you read.
\\

\begin{cuenotes}
  \cue{Know your purpose}
  \note{
    Yes, you're reading because your professor told you to do so, but there is
    more to it than that. What will you be asked to do with the information you
    gather from your reading assignment? Reading in preparation for a
    multiple-choice exam requires a greater attention to detail (think keywords,
    definitions, dates and specific concepts and examples) than reading to
    prepare for discussion or to write an essay (think main points and
    relationships). Consider your purpose for reading and what you need to be
    able to understand, know, or do after reading. Keep this purpose in mind as
    you read.
  }
  \cue{Integrate prior knowledge}
  \note{
    You already know so much; why not help yourself out? Before previewing the
    text, determine what you already know about the material you are to read.
    Think about how the reading relates to other course topics, and ask why your
    professor might have assigned the text. Identify personal experiences or
    second-hand knowledge that relates to the topic. Make a list of things you
    want to know about the text or questions that you want to try to answer
    while reading.
  }
  \cue{Preview the text}
  \note{
    Don't jump in all at once. Give the text an initial glance, noting headings,
    diagrams, tables, pictures, bolded words, summaries, and key questions.
    Consider reading introductions and conclusions to gather main ideas. After
    you preview, predict what the section or chapter will be about and what the
    main concepts are going to be.
  }
  \cue{Plan to break your reading into manageable chunks}
  \note{
    Do you have five days to read twenty pages? Read four pages a night. Twenty
    pages in only one night? Read four pages and then take a fifteen-minute
    break to rest your mind and move your body. Taking breaks while reading
    improves focus, motivation, understanding, and retention. Plus, it's
    healthier for our bodies! Try using a weekly calendar or the Pomodoro
    Technique to break up and schedule your time.
  }
  \cue{Decide whether and how to read from a screen}
  \note{
    Especially if you are taking courses online or studying remotely, some of
    your course materials may be in a digital format, such as online journal
    articles or electronic textbooks. Before you read, decide if your reading is
    something you could and would want to print out. Sometimes it is easier to
    grasp content when it is on paper. If this is not your preference or is not
    an option, make reading breaks an even higher priority, consider adjusting
    your screen, and be strategic about the time of day when you are reading in
    order to avoid eye strain or headaches.
  }
\end{cuenotes}
%
\section{While reading}
Keeping your brain active and engaged while you read decreases distractions,
mind-wandering, and confusion. Try some of these strategies to keep yourself
focused on the text and engaged in critical thinking about the text while you
read.
\\

\begin{cuenotes}
  \cue{Self-monitor}
  \note{
    The only one who can make sure you're engaged while reading is you! If you
    are able to think about what you will eat for dinner or what will happen
    next on that Netflix show you love, you are no longer paying attention! As
    soon as you notice your mind drifting, STOP and consider your needs. Do you
    need a break? Do you need a more active way to engage with the text? Do you
    need background noise or movement? Do you need to hear the text aloud? What
    about a change of environment? Before resuming, summarize the last chunk of
    text you remember to make sure that you know the appropriate starting point.
  }
  \cue{Annotate}
  \note{
    Overusing the highlighter? Put it down and try annotation. Develop a
    key/system to note the following in the text: key ideas/major points,
    unfamiliar words/unclear information, key words and phrases, important
    information, and connections.
  }
  \cue{Summarize}
  \note{
    After reading small sections of texts (a couple of paragraphs, a page, or a
    chunk of text separated by a heading or subheading), summarize the main
    points and two or three key details in your own words. These summaries can
    serve as the base for your notes while reading.
  }
  \cue{Ask hard questions}
  \note{
    Think like a professor and ask yourself higher level, critical thinking
    questions, such as:
    \begin{itemize}
      \item What differences exist between $ \ldots $?
      \item How is $ \ldots $ an example of $ \ldots $?
      \item What evidence can you present for $ \ldots $?
      \item What are the features of $ \ldots $?
      \item What would you predict from $ \ldots $?
      \item What solutions would you suggest for $ \ldots $?
      \item Do you agree that $ \ldots $? Explain.
      \item What is the most important feature of $ \ldots $?
      \item How is the text guiding the reader to come to certain conclusions?
      \item Who is the intended audience?
      \item What premises or prior knowledge does the text require to make its
        argument(s)?
    \end{itemize}
  }
\end{cuenotes}
%
\section{After reading}
Reading a text should not end at the end of the chapter. Using effective after
reading strategies can help you better understand and remember the text
long-term.
\\

\begin{cuenotes}
  \cue{Check in with yourself}
  \note{
    Whether you read a printed text or an online document, the most important
    thing to assess is how much you understood from your reading. This
    \textbf{metacognitive skill} is one of the hardest to practice because if
    you truly missed the mark on what you read, you might not know until you get
    to class---or worse, until test day.
  }
  \note{
    Here are some ways to self-check your reading comprehension. Try
    ``cross-referencing'' the information you read with simpler writings on the
    same subject and discussing your takeaways with peers. If you and your peers
    vary widely in your takeaways, go back to the text to see if the
    presentation of evidence can account for these discrepancies. Some key
    questions:
    \begin{itemize}
      \item Are there multiple possible ``answers'' here?
      \item Is there a blind spot in your knowledge on the subject?
      \item Is the language of the text too difficult or unclear?
      \item Are different sources on the same topic using consistent language,
        or are they using different language to discuss the same or similar
        things?
    \end{itemize}
  }
  \cue{Show what you know}
  \note{
    \begin{itemize}
      \item Create an outline of the text from memory, starting with the main
        points and working toward details, leaving gaps when necessary to go
        back to the text for facts or other things you can't remember.
      \item Discuss the material with a friend or classmate.
      \item Call a family member and teach them what you now know.
      \item Brain dump: write down everything you remember from the reading in 5
        minutes.
      \item Ask yourself critical questions about the reading and answer those
        questions in a timed format.
      \item Identify the important concepts from the reading and provide examples and non-examples of each concept.
      \item Create a concept map from memory to illustrate your learning from the assigned reading.
      \item Take screenshots from digital texts as a starting point for class notes or annotations.
    \end{itemize}
  }
  \cue{Investigate further}
  \note{
    If any information remains unclear, locate other resources related to the
    topic such as a trusted video source or web-based study guide. Still have
    questions you can't answer on your own? Make note of them to ask a professor,
    TA, or classmate.
  }
  \cue{Self-test}
  \note{
    Create flashcards or an outline for the main concepts, terms, dates, etc. in
    the text. Use the flashcards or outline to test yourself on what you read and
    see how much you remember and can explain correctly. Cover the answers or
    explanations and don't look at them until after you have already answered or
    explained in your own words. Pause videos periodically and use your own
    knowledge to supply an answer or predict where the video is going. Then hit
    play to see if you are on track.
  }
  \note{
    Self-testing in this way will help you synthesize and think through the
    information and recall it better in the future.
  }
\end{cuenotes}
%
\end{document}
