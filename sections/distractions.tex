\documentclass[../main.tex]{subfiles}
\graphicspath{{\subfix{../figures/}}}
%
\title{Distractions}
%
\begin{document}
\maketitle
%
\section{Introduction}
Let’s be honest: we all struggle with distractions to some degree. Distractions
can take many forms, including our phones, computers, friends, or our own
thoughts. In college, distractions can be even more abundant than in high
school, because there are so many new opportunities and experiences available.
Additionally, most college students have more flexibility and less structure in
college than they did in high school. Whether you’re learning remotely or living
on campus, you may have long periods of unstructured time when you will have to
decide how to use your time wisely. Usually, no one else is there to keep you on
task—you’re in charge of making your own schedule and focusing when it’s time to
study.

Many students struggle to stay focused and end up not getting the most out of
their study sessions; then they sometimes find they need to cram at the last
minute to get work finished. Fortunately, there are many strategies available to
keep yourself distraction free. This handout shares strategies to manage
internal and external distractions so that you can maximize your focus (and your
success) in college.
%
\section{Managing internal distractions}
Internal distractions are your own thoughts and emotions. These can include
thoughts about pressing responsibilities or pleasant things that you’d rather be
doing. This can also include emotions about life circumstances, the task you are
working on, fears, and worries. Circumstances like major world events and
personal struggles can be sources of internal distractions. Below are some tips
to help you manage your internal distractions.
\\

\begin{cuenotes}
  \cue{Make a daily plan}
  \note{
    \begin{itemize}
      \item Schedule time for each task that you have to do. Plan to work in
        short chunks (no more than one hour at a time) and then take a break!
        Incorporating breaks will help you stay focused during your work time.
      \item Incorporate a change of scenery. Take a break by going on a short
        walk around your neighborhood or apartment complex.
      \item Discover the best time of day for you to tackle challenging
        assignments. Doing the most challenging tasks first thing in the morning
        can help prevent getting caught up with distractions, but do what works
        best for you.
      \item Discover where you study best. Does working in your bed make you
        tired? Try studying somewhere you designate exclusively for work, like a
        desk, a comfy chair in your house, a coffee shop, or the library.
      \item Incorporate movement and fun. Make sure to schedule in times to
        participate in activities that you enjoy each day and week. Add some
        movement or exercise into your daily schedule. This could mean taking a
        fitness class at the gym, going on a run, following an exercise video
        online, or dancing. Movement while studying can also help you stay
        focused. Try a treadmill desk at the Student Union or use a cardboard
        box to turn your desk into a standing desk. How about using a white
        board?
    \end{itemize}
  }
  \cue{Manage your thoughts while studying}
  \note{
    \begin{itemize}
      \item Plan an activity to transition your mind for focus, like deep
        breathing or listening to music.
      \item Write down competing and distracting thoughts on a post-it or
        notebook and save them for later. This way, you won’t forget about them
        but you hopefully will be able to put them aside until you are done
        working.
      \item Consider building movement into your study time.
    \end{itemize}
  }
  \cue{Get enough rest!}
  \note{
    Everyone is more distracted when tired. Aim for 7-9 hours of sleep each
    night.
  }
  \cue{Set SMART goals (specific, measurable, achievable, realistic, timely)}
  \note{
    Having specific goals can help you stay on task and feel motivated.
  }
  \cue{Engage in self-talk}
  \note{
    \begin{itemize}
      \item If you find your mind wandering when you should be working, tell
        yourself to get back on task and that you need to complete this work.
      \item Praise yourself and verbally reinforce positive behaviors. Tell
        yourself that you did a great job when you accomplish a task.
      \item Remind yourself that you are capable and just need to put forth more
        effort if you start thinking you don’t have what it takes to succeed.
      \item Engage in self care and be kind to yourself. Make sure your goals
        are achievable and realistic. Focus on progress and growth. Remember
        that during challenging times, it can be easier to start with small
        goals and build from there.
    \end{itemize}
  }
  \cue{Practice self-regulation}
  \note{
    Self-regulation is when you use processes to be aware of and control your
    behaviors and thoughts. This will help you deal with distractions that can
    interfere with your learning. For example, move to another table when you
    are in the library and distracted by someone talking near you. If studying
    in your living room tempts you to talk with your roommates or your family,
    consider working in your bedroom.
  }
\end{cuenotes}
%
\section{Managing external distractions}
External distractions are ones that originate outside of you—things like
technology (phones, social media, websites, YouTube, video games, Netflix),
other people, or noises around you. Below are some tips for managing external
distractions.
\\

\begin{cuenotes}
  \cue{Pick a setting that is a good match for the academic task.}
  \note{
    \begin{itemize}
      \item Can you really stay focused in your dorm room or house when
        studying?
      \item How can you manage distractions within your home? Are there certain
        rooms that are quieter than others? WHich areas of your home have the
        most foot traffic?
      \item What’s better: a group setting or working alone?
      \item What’s better: the library or a cozy spot in a coffee shop?
      \item Consider the noise level you need to work productively:
        \begin{enumerate}
          \item Do you work better with complete silence or a little background
            noise?
          \item Do you need earplugs or head phones to cancel out surrounding
            noise?
          \item Try background sound. Play white noise on your computer, like
            rainymood, coffitivity, or simplynoise. Run a fan or play quiet
            music.
          \item Can you coordinate with your roommates or family to create
            “quiet zones” in certain spaces of your home depending on the time
            of day?
        \end{enumerate}
    \end{itemize}
  }
  \cue{Seek accountability}
  \note{
    Ask a friend, roommate, or classmate to keep you accountable to your goals
    and fight against distractions. Here are some ways a friend can help:
    \begin{itemize}
      \item Give your phone or laptop to a friend or roommate to hold onto when
        you are studying.
      \item Try studying with a friend or group to hold each other accountable
        to staying on task.
      \item Try studying virtually with a friend or classmate through video chat
        to help both of you stay accountable.
    \end{itemize}
  }
  \cue{Take charge of technology distractions}
  \note{
    Limit or bar yourself from unnecessary technology use during study and class
    times. This is another thing you can ask a friend to hold you accountable
    to!
    \begin{itemize}
      \item Leave your smartphone, laptop, etc. at home, in another room, or
        with a friend while studying.
      \item Use internet-blocking sites or self-management tools.
    \end{itemize}
  }
\end{cuenotes}
%
\end{document}
