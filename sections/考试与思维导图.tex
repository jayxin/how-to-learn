\documentclass[../main.tex]{subfiles}
\graphicspath{{\subfix{../figures/}}}
%
\title{考试与思维导图}
%
\begin{document}
%
\maketitle
%
\section{思维导图也可以搞定考试}
%
\subsection{我的经历}
\begin{cuenotes}
    \cue{结果}
    \note{借助思维导图的复习, 顺利通过考试}
    \cue{方法}
    \note{提炼梳理-结构化要点-联想记忆}
\end{cuenotes}
%
\subsection{思维导图}
\begin{cuenotes}
    \cue{适合记忆}
    \note{
      大脑特点:
      \begin{itemize}
        \item 更喜欢记忆简洁的信息
        \item 习惯性将不完整信息补全
        \item 丰富的颜色图画促进记忆
      \end{itemize}
    }
    \cue{信息整合}
    \note{
      \begin{itemize}
        \item 提取关键词, 促进主动思考
        \item 联想整合, 加入自己的理解
      \end{itemize}
    }
\end{cuenotes}

因为思维导图是由关键字的形式呈现的,
我们会有一种想要去把各个知识点给补充完整的冲动,正是这样的冲动,
促进我们建立大脑中的记忆链接。

其次是因为,作图的过程就是``信息整合''的过程。

在我们制作思维导图的过程中,一方面要对关键词进行提取,
这就需要我们在总结的过程中需要不断向自己提问,化被动吸收为主动思考。
另一方面要进行整合联想,在新旧知识之间建立联系,加入自己的理解。

相对于干巴巴的纯文字笔记,这种思考、分析、理解的过程,
可以有效促进主动学习,非常适合进行信息整合和记录笔记的场景。
%
\section{如何制作知识点思维导图}
\begin{cuenotes}
  \cue{信息类型}
  \note{
    \begin{itemize}
      \item 具体信息: 历史政治这类以记忆为主的知识(适合制作思维导图记忆)
      \item 抽象信息: 数学物理这类理解计算为主的知识
    \end{itemize}
  }
  \cue{抓住重点}
  \note{
    \begin{itemize}
      \item 有参考: 按老师归纳的重点章节和知识点整理复习
      \item 无参考: 分析真题出题位置和频率, 找到考试重点
    \end{itemize}
  }
\end{cuenotes}

并不是要把整本书的内容都做到图中,我们记忆的目的是为了考试得分,
在时间有限的情况下,我们要先抓住重点、易考点的记忆。
剩下的特别细碎的知识点,推荐大家用碎片时间反复记忆。

教科书里的内容结构性是非常强的,给我们制图带来了很多方便,
主要就是分清主次顺序,使用关键词去概括知识点的主要内容。

也因此,选择关键词很重要,太长了不便于记忆,
不合理又不方便后续我们联想书中内容。我们要注意以下几点:
\begin{itemize}
  \item 首先是抓住\emph{词性},名词为主,动词次之,
    辅以必要的修饰词。简单来说,只要不影响意思,
    修饰词、连接词和部分名词动词都可以删除。
  \item 其次是控制\emph{字数},思维导图的关键字由二到五个字最好,
    概括知识点的时候可以用短句,但一般也不要超过十个字。
  \item 最后就是使用具有\emph{明确含义}的词,
    能够让你联想到特定场景/语句的词,方便后续记忆与答题。
\end{itemize}
%
\section{怎么用思维导图信息答题}
\begin{cuenotes}
  \cue{记忆与展开}
  \note{
    \begin{enumerate}
      \item 主动回忆
        \begin{itemize}
          \item 课文读很多遍, 不如尝试被一遍效果好
          \item 根据导图内容, 回忆知识点, 尝试记忆
          \item 回忆不起来的, 回到书中位置进行复习
        \end{itemize}
      \item 考试答题
        \begin{itemize}
          \item 脑海中建立思维导图的整体框架
          \item 看到题目$ \rightarrow $联想知识点$ \rightarrow $
            在导图的位置$ \rightarrow $想到了顺着分支往下走
        \end{itemize}
    \end{enumerate}
  }
  \cue{打印记忆}
  \note{
    \begin{itemize}
      \item 导图设置:
        \begin{enumerate}
          \item 字体大小 20 以上
          \item 结构控制: 竖版(portrait)就尽量向下做, 横版(landscape)就尽量向两边发散
        \end{enumerate}
      \item PDF 导出
    \end{itemize}
  }
\end{cuenotes}
%
\subsection{如何展开被压缩的信息}
有人可能会有疑惑,把信息压缩了,那我到考试的时候写论述题怎么展开呢?

第一是思维补全。人类在阅读不完整的内容时,大脑会不自觉的补充信息进去。
正是这样的冲动帮助我们答题。

第二是主动回忆。小时候有这样的经历,课文读了很多遍,
不如尝试背一遍来的效果好,这是主动回忆的力量。
我们用思维导图记忆就是用的这个方法。

所以,在之后的复习过程中,我是这样运用思维导图的:
\begin{itemize}
  \item 首先,根据图中的内容,不断去回忆书中的知识点,尝试用自己的话把它说下来。
  \item 对于实在回忆不起来的内容,就去书上看一下。慢慢地随着熟练,
    脑海中对于这张图就越来越清晰,直到能回忆起大部分导图信息,
    顺着导图的连接线就能想到下面的内容。
\end{itemize}

这就是我后来考试背诵时的状态:
考场上看到题,立刻联想知识点$\rightarrow$在思维导图的哪个位置$\rightarrow$
想到了接着顺着分支往下走,所有的知识点就被顺藤摸瓜了。
%
\end{document}
